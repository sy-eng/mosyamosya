\documentclass{jsarticle}
\usepackage{amsmath,amssymb,amsfonts}

\begin{document}
別証のI)について検討する。

証明したいことはn次元のr個のベクトルのr-ベクトルに対して、強い意味で一次独立($|a_1, \hdots, a_r| \neq \bf{0}$)のとき、
その一部をi個除いたものの(r-i)-ベクトルも強い意味で独立である((r-1)-ベクトルも$\bf{0}$でない)こと。
(例えば、$|a_1, \hdots, a_{r-1}| \neq \bf{0}$)

まず、上記のベクトル$a_1, \hdots, a_r$を並べた(n, r)行列を以下のように置く。
\begin{equation}
	\begin{pmatrix}
		a_{11} & a_{12} & \hdots & a_{1r} \\
		a_{21} & a_{22} & \hdots & a_{2r} \\
		\vdots & \vdots & \hdots & \vdots \\
		a_{i1} & a_{i2} & \hdots & a_{ir} \\
		\vdots & \vdots & \hdots & \vdots \\
		a_{n1} & a_{n2} & \hdots & a_{nr} \\
	\end{pmatrix}
\end{equation}
仮定より、r-ベクトルは0でないから、ある組み合わせ、$\nu = (\alpha_1^{(\nu)}, \alpha_2^{(\nu)}, \hdots, \alpha_r^{(\nu)})$に関して、
\begin{equation}
	\begin{vmatrix}
		a_{\alpha_1^{(\nu)}1} & a_{\alpha_1^{(\nu)}2} & \hdots & a_{\alpha_1^{(\nu)}r} \\
		a_{\alpha_2^{(\nu)}1} & a_{\alpha_2^{(\nu)}2} & \hdots & a_{\alpha_2^{(\nu)}r} \\
		\vdots & \vdots & \hdots & \vdots \\
		a_{\alpha_r^{(\nu)}1} & a_{\alpha_r^{(\nu)}2} & \hdots & a_{\alpha_r^{(\nu)}r} \\
	\end{vmatrix} \neq 0
\end{equation}
この左辺はIIの定理5の展開定理より、例えば最後の列($a_r$の要素)に関して、展開でき、
\begin{equation}
	\label{det}
	|A^{(\nu)}| = 
	\begin{vmatrix}
		a_{\alpha_1^{(\nu)}1} & a_{\alpha_1^{(\nu)}2} & \hdots & a_{\alpha_1^{(\nu)}r} \\
		a_{\alpha_2^{(\nu)}1} & a_{\alpha_2^{(\nu)}2} & \hdots & a_{\alpha_2^{(\nu)}r} \\
		\vdots & \vdots & \hdots & \vdots \\
		a_{\alpha_r^{(\nu)}1} & a_{\alpha_r^{(\nu)}2} & \hdots & a_{\alpha_r^{(\nu)}r} \\
	\end{vmatrix}
	=
	a_{\alpha_1^{(\nu)}r}\Delta_{\alpha_1^{(\nu)}r} + a_{\alpha_2^{(\nu)}r}\Delta_{\alpha_2^{(\nu)}r} + \hdots + a_{\alpha_r^{(\nu)}r}\Delta_{\alpha_r^{(\nu)}r}
	\neq 0
\end{equation}
ここで、
\begin{equation}
\Delta_{\alpha_i^{(\nu)}r} = 
	\begin{vmatrix}
		a_{\alpha_1^{(\nu)}1} & a_{\alpha_1^{(\nu)}2} & \hdots & a_{\alpha_1^{(\nu)}(r-1)} \\
		a_{\alpha_2^{(\nu)}1} & a_{\alpha_2^{(\nu)}2} & \hdots & a_{\alpha_2^{(\nu)}(r-1)} \\
		\vdots & \vdots & \hdots & \vdots \\
		a_{\alpha_{i-1}^{(\nu)}1} & a_{\alpha_{i-1}^{(\nu)}2} & \hdots & a_{\alpha_{i-1}^{(\nu)}(r-1)} \\
		a_{\alpha_{i+1}^{(\nu)}1} & a_{\alpha_{i+1}^{(\nu)}2} & \hdots & a_{\alpha_{i+1}^{(\nu)}(r-1)} \\
		\vdots & \vdots & \hdots & \vdots \\
		a_{\alpha_r^{(\nu)}1} & a_{\alpha_r^{(\nu)}2} & \hdots & a_{\alpha_r^{(\nu)}(r-1)} \\
	\end{vmatrix}
\end{equation}
この$\Delta_{\alpha_i^{(\nu)}r}$を見てみると、$a_1, \hdots, a_{r-1}$のある行の組み合わせ
$\nu' = (\alpha_1^{(\nu)}, \alpha_2^{(\nu)}, \hdots, \alpha_{(i-1)}^{(\nu)}, \alpha_{(i+1)}^{(\nu)}, \hdots,\alpha_r^{(\nu)})$
に対して、行列式をとっているので、$\Delta_{\alpha_i^{(\nu)}r}$は$|a_1, \hdots, a_{r-1}|$の要素になっている。
(\ref{det})を考えると、行列式が0にならない場合、ある$1 \leq p \leq r$があって、$\Delta_{\alpha_p^{(\nu)}r} \neq 0$。
(そうでないと、$|A^{(\nu)}| = 0$となってしまう。)

よって、$|a_1, \hdots, a_{r-1}|$のある要素は0でないので、$|a_1, \hdots, a_{r-1}| \neq \bf{0}$

以上により、$a_1, \hdots, a_{r-1}, a_r$が強い意味で一次独立であれば、$a_1, \hdots, a_{r-1}$も強い意味で独立であることが示された。
ここで$a_r$を取り除いたが、$a_1, \hdots, a_{r-1}, a_r$のうち任意の1個を除いても同じことが言える。また、これを繰り返し適用することも
できるため、その一部のベクトルも強い意味で一次独立となる。
\end{document}
