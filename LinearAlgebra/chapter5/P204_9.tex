\documentclass{jsarticle}
\usepackage{amsmath,amssymb,amsfonts}

\begin{document}
P204の(9)の証明はその下に記載されている。これが定理6の証明と似たものとなっている。

P204から205の内容を確認してみる。

これらはP108の証明と同様に対応できる。

まず、(10)に関しては、P.204の最後のパラグラフから記載があるが、P108の(10)の証明と同様になっていることが確認できる。

P.205の$(W_1 + W_2)^{\perp} = {W_1}^{\perp} \cap {W_2}^{\perp}$を考える。

$W_1 + W_2 \supset {W_1}$なので、$(W_1 + W_2)^{\perp} \subset {W_1}^{\perp}$となる。これは任意の$a \in {(W_1 + W2)}^{\perp} \subset V^*$を考えると、すべての$b \in W_1$に対して、$<a, b> = 0$になる。よって、$a \in {W_1}^{\perp}$となり、上記が成り立つ。

同様に$(W_1 + W_2)^{\perp} \subset {W_2}^{\perp}$となり、$(W_1 + W_2)^{\perp} \subset {W_1}^{\perp} \cap {W_2}^{\perp}$

逆に、任意の$y \in {W_1}^{\perp} \cap {W_2}^{\perp} \in V^*$とすれば、$x = x_1 + x_2 \in W_1 + W_2 \in V, x_1 \in W_1 \subset V, x_2 \in W_2 \subset V$に対し、$<x, y> = <x_1, y> + <x_2, y> = 0$。よって、$y \in (W_1 + W_2)^{\perp}$。
すなわち、$(W_1 + W_2)^{\perp} \supset {W_1}^{\perp} \cap {W_2}^{\perp}$

故に$(W_1 + W_2)^{\perp} = {W_1}^{\perp} \cap {W_2}^{\perp}$

P.205の$(W_1 \cap W_2)^{\perp} = {W_1}^{\perp} + {W_2}^{\perp}$を考える。

これは空間が異なるので、P.108と同様の検討はできない。

$W_1 \cap W_2 \subset W_1$なので、$(W_1 \cap W_2)^{\perp} \supset {W_1}^{\perp}$となる。これは任意の$a \in {W_1}^{\perp} \subset V^*$を考えると、すべての$b \in W_1 \cap W_2$に対して、$<a, b> = 0$になる。よって、$a \in {(W_1 \cap W_2)}^{\perp}$となり、上記が成り立つ。

同様に$(W_1 \cap W_2)^{\perp} \supset {W_2}^{\perp}$となり、$(W_1 \cap W_2)^{\perp} \supset {W_1}^{\perp} + {W_2}^{\perp}$

逆に、任意の$y \in {W_1}^{\perp} + {W_2}^{\perp} \in V^*$とすれば、$x \in W_1 \cap W_2$に対し、$<x, y> = 0$。よって、$x \in (W_1 \cap W_2)^{\perp}$すなわち、$(W_1 \cap W_2)^{\perp} \subset {W_1}^{\perp} + {W_2}^{\perp}$。

故に、$(W_1 \cap W_2)^{\perp} = {W_1}^{\perp} + {W_2}^{\perp}$となる。
\end{document}
