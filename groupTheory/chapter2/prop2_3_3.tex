\documentclass{jsarticle}
\usepackage{amsmath,amssymb,amsfonts}

\begin{document}
省略された命題2.3.3の証明を書いてみる。

方針としては命題2.3.2を利用する。

まず、命題2.3.2を使うための条件として、$H_1 \cap H_2$がGの部分集合であることを示す。

$H_1$(もしくは$H_2$)はGの部分群であるため、$H_1 \cap H_2 \subset H_1 \subset G$となるため、$H_1 \cap H_2$はGの部分集合となる。

命題2.3.2の(1)-(3)を満たすと、$H_1 \cap H_2$がGの部分群であることが言える。

(1)について、$H_1,H_2$がGの部分群であるため、$I_G \in H_1$かつ、$I_G \in H_2$。
よって、$I_G \in H_1 \cap H_2$。\\
そのため、(1)が満たされる。

(2)について、$H_1,H_2$がGの部分群であるため、すべての、$x, y \in H_1 \cap H_2$に対して、$xy \in H_1$($H_1$が部分群であるため、$H_1 \cap H_2 \subset H_1$の要素、x,yの積xyは同じく命題2.3.2の(2)により、$H_1$の要素になる。)でかつ、同様に、$xy \in H_2$になる。つまり、$xy \in H_1 \cap H_2$となり、(2)が満たされる。

(3)について、ほぼ、(2)を示したときと同様に、すべての、$x \in H_1 \cap H_2$に対して、$x^{-1} \in H_1$であり、$x^{-1} \in H_2$になる。
つまり、$x^{-1} \in H_1 \cap H_2$となり、(3)が満たされる。

(1)-(3)が満たされたため、$H_1 \cap H_2$がGの部分群であることが言える。
\end{document}
