
\documentclass{jsarticle}
\usepackage{amsmath,amssymb,amsfonts}

\begin{document}
定理2.10.2に関して、証明を詳しく確認する。

\begin{quote}
定理2.10.2(準同型定理(部分群の対応))\\
1, Nを群Gの正規部分群、\\
2, $\pi : G \rightarrow G/N$を自然な準同型とする。\\
3, G/Nの部分群の集合を$\mathbb{X}$,\\ 
4, GのNを含む部分群の集合を$\mathbb{Y}$とするとき、\\
5, 写像
$\phi : \mathbb{X} \ni H \mapsto \pi^{-1}(H) \in \mathbb{Y}$, \\
6, $\psi : \mathbb{Y} \ni K \mapsto \pi(K) \in \mathbb{X}$\\
7, は互いに逆写像である。\\
8, したがって、集合$\mathbb{X}, \mathbb{Y}$は1対1に対応する。
\end{quote}

1に関して、Nが正規部分群だとすると、2.8にある性質たちが利用できる。

定理2.10.1にもあるが、上記の2について、定義2.6.9(1)を踏まえると、
$G \rightarrow G/N$自然な写像になる。自然な写像なので、
$\pi$が存在することは、明らか。命題2.8.13より、その写像は全射準同型で、
$Ker(\pi) = N$となる。

$\pi$は存在して、HがG/Nの部分群なので、その逆像は必ず存在する。
しかし、この写像$\phi$(上記、5にある。)が$\mathbb{Y}$への写像になっているか
(well-definedになるか)は確認する必要がある。これは、本の証明、1-4行目にある。

$H \in \mathbb{X}$なら、上記、3にあるようにHはG/Nの部分群なので、
命題2.3.2より、$1_{G/N} \in H$となる。よって、$\pi^{-1}(1_{G/N}) \subset \pi^{-1}(H)$となる。
($\pi^{-1}(1_{G/N})$は逆像で、複数の元を持つので、$\in$ではなく、$\subset$になる。)
上記の2に関する記述と、定義2.5.1(3)より、$\pi(h) = 1_{G/N}(h \in Ker(\pi) = N$であり、
$\pi^{-1}(1_{G/N}) = Ker(\pi) = N$となる。つまり、$N = \pi^{-1}(1_{G/N}) \subset \pi^{-1}(H)$となり、
$\mathbb{Y}$の条件になる、$\pi^{-1}(H)$がNを含むことを確認できた。
あとは、$\pi^{-1}(H)$がGの部分群であることが確認できれば良い。

部分群であることは命題2.3.2(1)-(3)を示せば良い。
(1)について、$N \subset \pi^{-1}(H)$であり、Nは正規部分群なので定義2.8.1,命題2.3.2(1)より、
$1_G \in N$になる。よって、$1_G = \pi^{-1}(H)$になる。
(2)について、$x,y \in \pi^{-1}(H) \subset G$なら、$\pi(x), \pi(y) \in H$になり、Hは群、$\pi$は
準同型なので、$\pi(x)\pi(y) = \pi(xy) \in H$となり、$xy \in \pi^{-1}(H)$になり、(2)も満たす。
(3)について、準同型、部分群の性質を考え、$x \in \pi^{-1}(H) \subset G$について、$x^{-1}$は
存在し、$\pi(x)\pi(x^{-1}) = \pi(xx^{-1}) = \pi(1_G) = 1_{G/N} \in H$になり、定理2.8.11より、
G/Nは部分群なので、$\pi(x^{-1}) \in H$。よって、$x^{-1} \in \pi^{-1}(H)$となり、(3)も満たす。
これにより、$\pi^{-1}(H)$がGの部分群であることも示され、$\phi$が$\mathbb{Y}$への写像に
なっていることが確認できた。

次に上記6にある、写像$\psi$が$\mathbb{X}$への写像になっているか
(well-definedになるか)は確認する必要がある。これは、本の証明、4-9行目にある。

$K \subset G$がNを含む部分群である時、$\pi(K)$が$G/N$の部分群であることを確認すればよい。
NがGの正規部分群なので定義2.8.1より、任意の$g \in G$に対して、$gNg^{-1} \in N$なので、
$g \in K \subset G$に対しても、当然成り立つ。よって、同様に定義2.8.1より、$N \triangleleft K$で
ある。K/NはKの元gによりgNという形をした剰余類の集合なので、G/Nの部分集合とみなすことができ、
また、$K/N = \pi(K)$(任意の$g \in K$に関して、gNという形をした剰余類を考えている。)である。
定理2.8.11より、K/Nは群になっているので、命題2.3.2より、Gの部分群になっている。
よって、$\pi(K) \in \mathbb{X}$となり、$\psi$は$\mathbb{X}$への写像になり、well-definedになる。

定義1.1.4(3)に従い、逆写像であることを確認する。
本の証明10-15行目では$\phi \circ \psi = id_{\mathbb{Y}}$つまり、任意の$K \in \mathbb{Y}$に
対して、$\phi \circ \psi(K) = K$であることを確認している。13-15行目にあるようにKはGの部分集合
(部分群)ではあるが、Kを$\mathbb{Y}$の元とみなしている。

本の10行目にあるようjに$H = \pi(K) $とおくと、$K \subset \pi^{-1}(H)$は明らかである。
($\pi$は写像なので、$k \in K$に対して、対応する$\pi(k) = j \in H$は必ず存在する。
ただし、ここまででは、$k' \notin K, \pi(k') = j' \in H$が存在しないことを言えていないので、
$K \subset \pi^{-1}(H)$となる。)$g \in \pi^{-1}(H)$とすると、この定義より、$\pi(g) \in \pi(K)$になる。
よって、ある$h \in K$があり、$\pi(g) = \pi(h)$となる。写像$\pi$と同値類の定義を考えるとG/Nで
同じ元になるので、$gN = hN$となる。つまりある$n_1 \in N$に対して、$g n_1 = h n_2$となる
$n_2 \in N$が存在する。Nは正規部分群なので$g = h n$となる$n \in N$が存在する。
$N \subset K$であり、Kは部分群なので、$g \in K$になる。これにより、任意の$g \in \pi^{-1}(H)$に
対して、$g \in K$が確認できたので、$\pi^{-1}(H) \subset K$になる。よって、$\pi^{-1}(H) = K$が確認でき、
任意のKに対して、$\phi \circ \psi(K) = \phi(\pi(K)) = \phi(H) = \pi(H) = K$となる。

次に、本の証明の16-19行目にあるように、$\psi \circ \phi = id_{\mathbb{X}}$つまり、任意の
$H \in \mathbb{X}$に対して、$\psi \circ \phi(H) = H$であることを確認して、上記と合わせて互いに
逆写像であることが言えるようになる。

$H \in \mathbb{X}$に対して、$\pi(\pi^{-1}(H)) \subset H$となる。($i \in H \subset G/N$に対して、
ある$s \in G$があり、$\pi(s) = h$だとすると、$i \in \pi(\pi^{-1}(\{i\}))$になるが、前記のようなsが
存在しないと、$i \notin \pi(\pi^{-1}(\{i\}))$になり、$\pi(\pi^{-1}(H)) \subset H$となる。)
定義2.8.13より、$\pi$は全射であるので、すべての$h \in H$に対して、$g \in G$があり、
$\pi(g) = h \in H$である。これは$g \in \pi^{-1}(H) (=G)$であることを意味する。
よって、$h = \pi(g) \in \pi(\pi^{-1}(H))$である。したがって、$H \subset \pi(\pi^{-1}(H))$となり、
結果として、$H = \pi(\pi^{-1}(H))$となる。これにより、$\psi \circ \phi(H) = \pi(\pi^{-1}(H)) = H$となる。

これにより、証明の10-19行目で、定義1.1.4(3)が示されたので、逆写像が存在することがわかった。

本のP.4にあるように、上記、7,8にあるような逆写像を持つこと、1対1に対応することは同値になる。
そのため、1対1対応することも言える。

\end{document}
                                  