\documentclass{jsarticle}
\usepackage{amsmath,amssymb,amsfonts}

\begin{document}
$f:A \rightarrow B$が写像なら、fが全単射であることと、fが逆写像を持つことが同値であることを言う。

まず、fが全単射であれば、fが逆写像を持つことを言う。

fが全単射であることは

\begin{itemize}
\item 単射である:$a, a' \in A, f(a) = f(a')$なら$a=a'$
\item 全射である:任意の$b \in B$に対し、$a \in A$があり、$f(a) = b$となる。
\end{itemize}

全射であるため、任意の$b \in B$に対し、$f(a) = b$となる、$a \in A$があるため、
bに対し、aを対応させることができる。

ある$b = f(a)$に対応する$a \in A$がaとa'のように複数あったとすると、単射の定義を外れるため、bに対応するaは唯一つで、上記のbのaへの対応が写像の定義を満たすため、この対応をBからAへの写像gと考えることができる。

すると、すべての$a \in A$に対し、$g \circ f(a) = g(f(a)) = g(b) = a$となるため、
$g \circ f = id_A$となり、同様に$f \circ g(b) = f(g(b)) = f(a) = b$となるため、
$f \circ g = id_B$となり、P.4の定義からfの逆写像、gが存在する。

次に、fが逆写像$g = f^{-1} : B \rightarrow A$を持つならば、全単射であることを言う。

背理法で単射であることを言う。

ある、$a, a' \in A, a \neq a'$に対し、$f(a) = f(a') = b \in B$だとする。
fに対して、逆写像gが存在し、a,a'に対して、$g(f(a)) = g(b) = a$, $g(f(a')) = g(b) = a'$となるが、この2式から$g(b) = a = a'$となるが、これは仮定に反する。

よって、逆写像を持つと単射である。

全射に関して、逆写像は$B \rightarrow A$の写像になっている。

そのため、任意の$b \in B$に対して、$g(b) = f^{-1}(b) = a \in A$が存在する。

逆写像なので、$f \circ f^{-1}(b) = f(a) = b$となるため、すべてのbに対して、$f(a) = b$となる、$a$が存在する。よって、全射であると言える。

単射でかつ、全射であるため、fが逆写像を持つと、全単射であると言える。

上記より、fが全単射であることと、fが逆写像を持つことは同値であると言える。
\end{document}
