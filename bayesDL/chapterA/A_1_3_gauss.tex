\documentclass{jsarticle}
\usepackage{amsmath,amssymb,amsfonts}

\begin{document}
A.1.3を確認する。
(A.14),(A.15)を踏まえると、(A.16)はxの関数と考えると以下のようになる。なお、共分散行列は対称行列であることに注意する。
\begin{equation}
\begin{split}
ln \, p(x|y) = ln \, \frac{p(x,y)}{p(y)} = ln \, p(x,y) + c = ln \, p(y|x)p(x) + c\\
=-\frac{1}{2}\{ (y - (Wx + b))^T {\Sigma_y}^{-1}(y - (Wx + b))\} -\frac{1}{2}\{ (x - \mu)^T {\Sigma_x}^{-1}(x - \mu)\} + c\\
=-\frac{1}{2}\{ x^T W^T {\Sigma_y}^{-1} W x + 2x^T W {\Sigma_y}^{-1} (y - b) \} -\frac{1}{2}\{ x^T {\Sigma_x}^{-1} x - 2 x^T {\Sigma_x}^{-1}\mu \} + c\\
=-\frac{1}{2}\{ x^T ({\Sigma_x}^{-1} + W^T {\Sigma_y}^{-1}W) x - 2 x^T ({\Sigma_x}^{-1}\mu + W {\Sigma_y}^{-1} (y - b)) \} + c\\
=-\frac{1}{2}\{ (x - ({\Sigma_x}^{-1} + W^T {\Sigma_y}^{-1}W)^{-1}({\Sigma_x}^{-1}\mu + W {\Sigma_y}^{-1} (y - b))^T ({\Sigma_x}^{-1} + W^T {\Sigma_y}^{-1}W) \\
(x - ({\Sigma_x}^{-1} + W^T {\Sigma_y}^{-1}W)^{-1}({\Sigma_x}^{-1}\mu + W {\Sigma_y}^{-1} (y - b)) \} + c\\
\end{split}
\end{equation}
この結果から確率$p(x|y)$はガウス分布となり、(A.17)から(A.19)がわかる。

ベイズの定理からわかる(A.20)も利用して、(A.21)を考える。
\begin{equation}
\begin{split}
ln \, p(y) = ln \, p(y|x) - ln \, p(x|y) + c
\end{split}
\end{equation}
yの関数だということに注意すると、
\begin{equation}
\begin{split}
ln \, p(y|x) = -\frac{1}{2} \{ (y - (Wx+b))^T {\Sigma_y}^{-1}(y - (Wx + b))\}+c 
= -\frac{1}{2} \{ y^T {\Sigma_y}^{-1} y - 2 y^T {\Sigma_y}^{-1} (Wx+b)\}+c
\end{split}
\end{equation}
\begin{equation}
\begin{split}
ln \, p(x|y) = -\frac{1}{2} \{ (x - \mu_{x|y})^T {\Sigma_{x|y}}^{-1}(x - \mu_{x|y})\}+c 
= -\frac{1}{2} \{ -2{\mu_{x|y}}^T {\Sigma_{x|y}}^{-1} x + {\mu_{x|y}}^T {\Sigma_{x|y}}^{-1} \mu_{x|y}\}+c\\
= -\frac{1}{2} \{ -2({\Sigma_x}^{-1}\mu + W^T {\Sigma_y}^{-1}(y-b))^T x + ({\Sigma_x}^{-1}\mu + W^T {\Sigma_y}^{-1}(y-b))^T {\Sigma_{x|y}} ({\Sigma_x}^{-1}\mu + W^T {\Sigma_y}^{-1}(y-b)) \}+c\\
= -\frac{1}{2} \{ y^T {\Sigma_y}^{-1}W\Sigma_{x|y} W^T{\Sigma_y}^{-1}y - 2y^T {\Sigma_y}^{-1} W(x + {\Sigma_{x|y}}(W^T{\Sigma_y}^{-1}b - {\Sigma_x}^{-1}\mu))\}+c\\
\end{split}
\end{equation}
よって、
\begin{equation}
\begin{split}
ln \, p(y) = ln \, p(y|x) - ln \, p(x|y) + c = -\frac{1}{2} \{ y^T {\Sigma_y}^{-1} y - 2 y^T {\Sigma_y}^{-1} (Wx+b)\} + \\
\frac{1}{2} \{ y^T {\Sigma_y}^{-1}W\Sigma_{x|y} W^T{\Sigma_y}^{-1}y - 2y^T {\Sigma_y}^{-1} W(x + {\Sigma_{x|y}}(W^T{\Sigma_y}^{-1}b - {\Sigma_x}^{-1}\mu))\} + c\\
= -\frac{1}{2} \{ y^T ({\Sigma_y}^{-1} - {\Sigma_y}^{-1}W\Sigma_{x|y} W^T{\Sigma_y}^{-1}) y - 2 y^T {\Sigma_y}^{-1} (b - W{\Sigma_{x|y}}(W^T{\Sigma_y}^{-1}b - {\Sigma_x}^{-1}\mu))\} + c\\
= -\frac{1}{2} \{ y^T ({\Sigma_y}^{-1} - {\Sigma_y}^{-1}W\Sigma_{x|y} W^T{\Sigma_y}^{-1}) y - 2 y^T {\Sigma_y}^{-1} (b + W{\Sigma_{x|y}}({\Sigma_x}^{-1}\mu - W^T{\Sigma_y}^{-1}b))\} + c\\
\end{split}
\end{equation}
ここで(A.1)を用いると、
\begin{equation}
A \equiv {\Sigma_y}^{-1} - {\Sigma_y}^{-1}W\Sigma_{x|y} W^T{\Sigma_y}^{-1} = {\Sigma_y}^{-1} - {\Sigma_y}^{-1}W ({\Sigma_x}^{-1} + W^T {\Sigma_y}^{-1}W)^{-1} W^T{\Sigma_y}^{-1}\\
= (\Sigma_y + W \Sigma_x W^T)^{-1}
\end{equation}
また、(A.2)を用いると、
\begin{equation}
\begin{split}
B \equiv A^{-1}{\Sigma_y}^{-1} (b + W{\Sigma_{x|y}}({\Sigma_x}^{-1}\mu - W^T{\Sigma_y}^{-1}b))\\
=(\Sigma_y + W \Sigma_x W^T){\Sigma_y}^{-1} (b + W({\Sigma_x}^{-1} + W^T {\Sigma_y}^{-1}W)^{-1}({\Sigma_x}^{-1}\mu - W^T{\Sigma_y}^{-1}b))\\
=(\Sigma_y + W \Sigma_x W^T) ({\Sigma_y}^{-1}b + {\Sigma_y}^{-1}W({\Sigma_x}^{-1} + W^T {\Sigma_y}^{-1}W)^{-1}({\Sigma_x}^{-1}\mu - W^T{\Sigma_y}^{-1}b))\\
=(\Sigma_y + W \Sigma_x W^T) ({\Sigma_y}^{-1}b + (\Sigma_y + W \Sigma_x W^T)^{-1} W \Sigma_x ({\Sigma_x}^{-1}\mu - W^T{\Sigma_y}^{-1}b))\\
=b + W \Sigma_x W^T {\Sigma_y}^{-1}b + W \Sigma_x ({\Sigma_x}^{-1}\mu - W^T{\Sigma_y}^{-1}b)
=b + W \Sigma_x W^T {\Sigma_y}^{-1}b + W \mu - W \Sigma_x W^T{\Sigma_y}^{-1}b\\
=W \mu + b
\end{split}
\end{equation}
\begin{equation}
\begin{split}
ln \, p(y) = -\frac{1}{2} \{ y^T A y - 2 y^T AB \} + c = -\frac{1}{2} \{ (y - B)^T A (y - B) \} + c \\
\end{split}
\end{equation}
よって、$p(y)$は(A.24)で表される。

さて、ここで$p(y_*| x_*, X, Y)$を考えると、(3.71)が(A.14)、(3.67)が(A.15)に対応するとして、(A.21)を考える。

すると、(A.14)のxがw,$\mu$が$\hat{\mu}$,$\Sigma_x$が$\hat{\Sigma}$に(A.15)のWが$\phi(x_*)^T$, bが0,$\Sigma_y$が$\sigma^{-2}$, yが$y_*$に対応する。
\begin{equation}
\begin{split}
ln \, p(y_* | x_*, Y, X) = -\frac{1}{2} \{ y_* ({\sigma_y}^{-2} - {\sigma_y}^{-2} \phi(x_*)^T (\hat{\Sigma}^{-1} + \phi(x_*)\sigma_y^{-2} \phi(x_*)^T)^{-1}\phi(x_*)\sigma_y^{-2}) y_* \\
-2 y_* {\sigma_y}^{-2} \phi(x_*)^T (\hat{\Sigma}^{-1} + \phi(x_*)\sigma_y^{-2} \phi(x_*)^T)^{-1} \hat{\Sigma}^{-1}\mu \} + c\\
= -\frac{1}{2} \{ ({\sigma_y}^{-2} - {\sigma_y}^{-4} \phi(x_*)^T (\phi(x_*)\sigma_y^{-2} \phi(x_*)^T + \hat{\Sigma}^{-1})^{-1}\phi(x_*)) {y_*}^{2} \\
-2 \phi(x_*)^T {\sigma_y}^{-2} (\phi(x_*)\sigma_y^{-2} \phi(x_*)^T + \hat{\Sigma}^{-1})^{-1} \hat{\Sigma}^{-1}\mu y_* \} + c
\end{split}
\end{equation}

この付近で誤植と思われるところをまとめておく。
\begin{itemize}
\item (A.2)に関して、行列P,Rは正定値行列でなく正則行列であれば、逆行列を持つので、(A.2)が成立すると思われる。PRMLでも(C.5)に同様の式があるが、特に条件は記載されていない。
\item ((A.3)は利用していないが、)(A.3)の右辺の全体には-1は不要と思われる。PRMLで(2.76)に同様の式がある。
\item (A.21)について、2番目の等号のあと、2項目の\{\}の中の2項目は-でなく+であると思われる。前後で、符号が一致しない。
\item (A.22)について、この式での式変形は(A.2)でなく、(A.1)を利用している。
\item (3.74)について、$y_*^2$の係数の$(\sigma_y^{-2}\phi(x_*)\phi(x_*)^T + \hat{\Sigma}^{-1})$は
$(\sigma_y^{-2}\phi(x_*)\phi(x_*)^T + \hat{\Sigma}^{-1})^{-1}$であると思われる。(A.21)に代入するとそうなる。
\end{itemize}
\end{document}
