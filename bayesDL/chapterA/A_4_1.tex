\documentclass{jsarticle}
\usepackage{amsmath,amssymb,amsfonts}

\begin{document}
A.4.1を確認する。

7.4.2の(7.87),(7.88)式にて、再帰的な定義が得られている。
しかし、この式には活性化関数として、一般的な非線形関数$\phi(a)$が
含まれているため、(7.89)で表される期待値の部分で解析的な計算ができない。

$\phi(a)$が特定の非線形関数のときは(7.89)の部分も解析的に求められ、
$\phi(a) = Erf(a)$のときは7.4.3.1に、$\phi(a) = \Theta(a)a^m$のときは、
7.4.3.2に示されている。

ここでは7.4.3.1に関して、詳細が記述されているA.4.1を確認する。

7.4.3にあるように、$a_1, a_2$は入力$x, x'$に対する活性であり、
それぞれ、スカラとなる。それを要素とする$a$は2次元のベクトルとなり、
(7.83)の上にあるようにそれは0を平均とする、ガウス分布と仮定しており、
その分散は2次元正方行列の$\Sigma$で表される。当然、これは対称行列になっている。

(7.90)にあるように活性化関数を
\begin{equation}
\phi(a) = Erf(a) = \frac{2}{\sqrt{\pi}} \int_0^a exp(-t^2)dt
\end{equation}
としたときを考える。

(A.45)のように、
\begin{equation}
\begin{split}
C = \mathbb{E}_{a \sim \mathcal{N}(0, \Sigma)}[\phi(a_1)\phi(a_2)] 
= \int \mathcal{N}(a | 0, \Sigma)Erf(a_1)Erf(a_2)da
= \int \mathcal{N}(a | 0, \Sigma)\phi(a_1)\phi(a_2)da\\
= \int (\frac{1}{(\sqrt{2\pi})^{2}\sqrt{|\Sigma|}} exp(-\frac{1}{2}a^T \Sigma^{-1} a)) \phi(e_1^T a)\phi(e_2^T a)da
= \frac{1}{2\pi |\Sigma|^{\frac{1}{2}}} \int exp(-\frac{1}{2}a^T \Sigma^{-1} a) \phi(e_1^T a)\phi(e_2^T a)da
\end{split}
\end{equation}
ただし、$e_1 = (1, 0)^T, e_2 = (0, 1)^T$。

$\Sigma$は共分散行列で、対称行列なので、コレスキー分解ができ、$\Sigma = L L^T$とおける。なおLは下三角行列になる。変数変換で$L\hat{a} = a$とし、$b_1^T = e_1^T L, b_2^T = e_2^T L$として、
\begin{equation}
\begin{split}
C = \frac{1}{2\pi ||L||L^T||^{\frac{1}{2}}} \int exp(-\frac{1}{2} \hat{a}^T L^T ({L^T}^{-1} L^{-1}) L \hat{a}) \phi(e_1^T L\hat{a})\phi(e_2^T L\hat{a}) (|L| d\hat{a})\\
= \frac{|L|}{2\pi ||L|^2|^{\frac{1}{2}}} \int exp(-\frac{1}{2} \hat{a}^T \hat{a}) \phi(e_1^T L\hat{a})\phi(e_2^T L\hat{a}) d\hat{a}
= \frac{1}{2\pi} \int exp(-\frac{1}{2} \hat{a}^T \hat{a}) \phi(b_1^T \hat{a})\phi(b_2^T \hat{a}) d\hat{a}
\end{split}
\end{equation}
(A.47)のように、補助変数を導入する。
\begin{equation}
C(\lambda) = \frac{1}{2\pi} \int exp(-\frac{1}{2} \hat{a}^T \hat{a}) \phi(\lambda b_1^T \hat{a})\phi(b_2^T \hat{a}) d\hat{a}
\end{equation}
ここで、$C(1) = C$。また、
\begin{equation}
Erf(0) = \frac{2}{\sqrt{\pi}} \int_0^0 exp(-t^2)dt = 0
\end{equation}
なので、
\begin{equation}
C(0) = \frac{1}{2\pi} \int exp(-\frac{1}{2} \hat{a}^T \hat{a}) \phi(0)\phi(b_2^T \hat{a}) d\hat{a}
= \frac{1}{2\pi} \int exp(-\frac{1}{2} \hat{a}^T \hat{a}) 0 \phi(b_2^T \hat{a}) d\hat{a}
= 0
\end{equation}
(A.49)を踏まえて、$C(\lambda)$を$\lambda$で微分すると、(A.48)のようになる。
\begin{equation}
\begin{split}
C'(\lambda) = \frac{1}{2\pi} \int exp(-\frac{1}{2} \hat{a}^T \hat{a}) \frac{d}{d\lambda}\phi(\lambda b_1^T \hat{a})\phi(b_2^T \hat{a}) d\hat{a}
= \frac{1}{2\pi} \int exp(-\frac{1}{2} \hat{a}^T \hat{a}) \frac{d}{d (\lambda b_1^T \hat{a})}\phi(\lambda b_1^T \hat{a}) \frac{d}{d \lambda}  \lambda b_1^T \hat{a} \phi(b_2^T \hat{a}) d\hat{a}\\
= \frac{1}{2\pi} \int exp(-\frac{1}{2} \hat{a}^T \hat{a}) \frac{2}{\sqrt{\pi}}exp(-(\lambda b_1^T \hat{a})^2) b_1^T \hat{a} \phi(b_2^T \hat{a}) d\hat{a}
= \frac{1}{2\pi} \frac{2}{\sqrt{\pi}} \int exp(-\frac{1}{2} \hat{a}^T \hat{a} - (\lambda^2 \hat{a}^T b_1 b_1^T \hat{a})) b_1^T \hat{a} \phi(b_2^T \hat{a}) d\hat{a}\\
= \frac{1}{{\pi}^{\frac{3}{2}}} \int exp(-\frac{1}{2} \hat{a}^T (I + 2\lambda^2 b_1 b_1^T) \hat{a}) b_1^T \hat{a} \phi(b_2^T \hat{a}) d\hat{a}
\end{split}
\end{equation}
\begin{equation}
(I + 2\lambda^2 b_1 b_1^T)^T = I + 2\lambda^2 b_1 b_1^T
\end{equation}
なので、$I + 2\lambda^2 b_1 b_1^T$は対称行列で、逆行列も対称行列となり、これもコレスキー分解ができる。
$(I + 2\lambda^2 b_1 b_1^T)^{-1} = L_b {L_b}^T$($L_b$は下三角行列)とし、変数変換$L_b \tilde{a} = \hat{a}$、$c_1^T = b_1^T L_b, c_2^T = b_2^T L_b$とすると、
\begin{equation}
\begin{split}
C'(\lambda) = \frac{1}{{\pi}^{\frac{3}{2}}} \int exp(-\frac{1}{2} \tilde{a}^T L_b^T {{L_b}^T}^{-1} L_b^{-1} L_b \tilde{a}) b_1^T L_b \tilde{a} \phi(b_2^T L_b \tilde{a}) |L_b| d\tilde{a}
= \frac{|L_b|}{{\pi}^{\frac{3}{2}}} \int exp(-\frac{1}{2} \tilde{a}^T \tilde{a}) c_1^T \tilde{a} \phi(c_2^T \tilde{a}) d\tilde{a}
\end{split}
\end{equation}
これにより、(A.50)が求まった。

(A.51)に関して、要素に分解してみる。
\begin{equation}
\begin{split}
C'(\lambda) = \frac{2|L_b|{c_1}^T c_2}{\pi^{2}} \int exp(-\frac{1}{2}\tilde{a}^T(I + 2c_2 {c_2}^T)\tilde{a})d\tilde{a}\\
= \frac{2|L_b|(c_{1, 1}c_{2, 1} + c_{1, 2}c_{2, 2})}{\pi^{2}} \int exp(-\frac{1}{2}\tilde{a}^T
\begin{pmatrix}
1 + 2c_{2, 1}^2 & 2c_{2, 1}c_{2, 2}\\
2c_{2, 1}c_{2, 2} & 1+2c_{2, 2}^2
\end{pmatrix}
\tilde{a})d\tilde{a}\\
= \frac{2|L_b|(c_{1, 1}c_{2, 1} + c_{1, 2}c_{2, 2})}{\pi^{2}} \int \int exp(-\frac{1}{2}(\tilde{a}_{1}^2(1 + 2c_{2, 1}^2) + 4\tilde{a}_{1}\tilde{a}_{2}c_{2, 1}c_{2, 2} + \tilde{a}_{2}^2(1 + 2c_{2, 2}^2))d\tilde{a}_1d\tilde{a}_2\\
= \frac{2|L_b|(c_{1, 1}c_{2, 1} + c_{1, 2}c_{2, 2})}{\pi^{2}} \int \int exp(-\frac{1}{2}\tilde{a}_{1}^2)exp(-\frac{1}{2}\tilde{a}_{2}^2)exp(-(c_{2, 1} \tilde{a}_{1} + c_{2, 2} \tilde{a}_{2})^2)d\tilde{a}_1d\tilde{a}_2
\end{split}
\end{equation}
これが、(A.50)と等しいことを確認する。

ここで、(2.28)を振り返る。$y = \sqrt{2} t$の変数変換を考えると、
\begin{equation}
\begin{split}
erf(\frac{2}{\sqrt{2}}) = \frac{2}{\sqrt{\pi}}\int_0^{\frac{x}{\sqrt{2}}} exp(-t^2) dt
= \frac{2}{\sqrt{2\pi}} \int_0^{x} exp(-\frac{y^2}{2}) dy\\
= \frac{2}{\sqrt{2\pi}} (\int_{-\infty}^x exp(-\frac{y^2}{2}) dy - \int_{-\infty}^0 exp(-\frac{y^2}{2}) dy)
= 2 (\int_{-\infty}^x \frac{1}{\sqrt{2\pi}} exp(-\frac{y^2}{2}) dy - \int_{-\infty}^0 \frac{1}{\sqrt{2\pi}} exp(-\frac{y^2}{2}) dy)\\
= 2 (\int_{-\infty}^x \mathcal{N}(y | 0, 1) dy - \int_{-\infty}^0 \mathcal{N}(y | 0, 1) dy)
= 2 (\Phi(x) - \Phi(0))
= 2 (\Phi(x) - \frac{1}{2})
\end{split}
\end{equation}
よって、(2.28)にあるように、
\begin{equation}
\Phi(x) = \frac{1}{2}(1 + erf(\frac{x}{\sqrt{2}}))
\end{equation}
$0 < \Phi(x) < 1$を考慮すると、$-\frac{1}{2} < erf(x) < \frac{1}{2}$となる。つまり、$erf(x)$は有限の値を取る。

(A.50)から変形する。
\begin{equation}
\begin{split}
C'(\lambda) = \frac{|L_b|}{{\pi}^{\frac{3}{2}}} \int exp(-\frac{1}{2} \tilde{a}^T \tilde{a}) c_1^T \tilde{a} \phi(c_2^T \tilde{a}) d\tilde{a}\\
= \frac{|L_b|}{{\pi}^{\frac{3}{2}}} \int \int exp(-\frac{1}{2} (\tilde{a}_1 \tilde{a}_1 + \tilde{a}_2 \tilde{a}_2)) (c_{1,1}\tilde{a}_1 + c_{1,2}\tilde{a}_2)\frac{2}{\sqrt{\pi}}\int_0^{c_{2,1} \tilde{a}_1 + c_{2,2} \tilde{a}_2} exp(-t^2) dt d\tilde{a}_1 d\tilde{a}_2\\
= \frac{2|L_b|}{{\pi}^{2}} (\int exp(-\frac{1}{2} \tilde{a}_2^2) \int c_{1,1}\tilde{a}_1 exp(-\frac{1}{2} \tilde{a}_1^2) \int_0^{c_{2,1} \tilde{a}_1 + c_{2,2} \tilde{a}_2} exp(-t^2) dt d\tilde{a}_1 d\tilde{a}_2\\
+ \int exp(-\frac{1}{2} \tilde{a}_1^2) \int c_{1,2}\tilde{a}_2 exp(-\frac{1}{2} \tilde{a}_2^2) \int_0^{c_{2,1} \tilde{a}_1 + c_{2,2} \tilde{a}_2} exp(-t^2) dt d\tilde{a}_2 d\tilde{a}_1)\\
= \frac{2|L_b|}{{\pi}^{2}} (\int exp(-\frac{1}{2} \tilde{a}_2^2) c_{1,1} ([- exp(-\frac{1}{2} \tilde{a}_1^2) \int_0^{c_{2,1} \tilde{a}_1 + c_{2,2} \tilde{a}_2} exp(-t^2) dt]_{-\infty}^{\infty} + \int exp(-\frac{1}{2} \tilde{a}_1^2) c_{2,1}\\
exp(-(c_{2,1} \tilde{a}_1 + c_{2,2} \tilde{a}_2)^2) d\tilde{a}_1)) d\tilde{a}_2
+ \int exp(-\frac{1}{2} \tilde{a}_1^2) c_{1,2} ([- exp(-\frac{1}{2} \tilde{a}_2^2) \int_0^{c_{2,1} \tilde{a}_1 + c_{2,2} \tilde{a}_2} exp(-t^2) dt]_{-\infty}^{\infty}\\
+ \int exp(-\frac{1}{2} \tilde{a}_2^2) c_{2,2}exp(-(c_{2,1} \tilde{a}_1 + c_{2,2} \tilde{a}_2)^2) d\tilde{a}_2) d\tilde{a}_1)\\
= \frac{2|L_b|}{{\pi}^{2}} (c_{1, 1}c_{2,1} \int exp(-\frac{1}{2} \tilde{a}_2^2) \int exp(-\frac{1}{2} \tilde{a}_1^2) exp(-(c_{2,1} \tilde{a}_1 + c_{2,2} \tilde{a}_2)^2) d\tilde{a}_1 d\tilde{a}_2\\
+ c_{1,2} c_{2,2} \int exp(-\frac{1}{2} \tilde{a}_1^2) \int exp(-\frac{1}{2} \tilde{a}_2^2) exp(-(c_{2,1} \tilde{a}_1 + c_{2,2} \tilde{a}_2)^2) d\tilde{a}_2 d\tilde{a}_1)\\
= \frac{2|L_b|(c_{1, 1}c_{2,1} + c_{1, 2}c_{2,2})}{{\pi}^{2}} \int \int exp(-\frac{1}{2} \tilde{a}_1^2) exp(-\frac{1}{2} \tilde{a}_2^2) exp(-(c_{2,1} \tilde{a}_1 + c_{2,2} \tilde{a}_2)^2) d\tilde{a}_1 d\tilde{a}_2
\end{split}
\end{equation}
となり、一致することがわかった。

$[- exp(-\frac{1}{2} \tilde{a}_2^2) \int_0^{c_{2,1} \tilde{a}_1 + c_{2,2} \tilde{a}_2} exp(-t^2) dt]_{-\infty}^{\infty}$に関して、積分部分は先に示したように有限の値になるので、0になることがわかる。
また、$\int_0^{c_{2,1} \tilde{a}_1 + c_{2,2} \tilde{a}_2} exp(-t^2) dt$の微分に関して、$(\Phi(c_{2,1} \tilde{a}_1 + c_{2,2} \tilde{a}_2) - \Phi(0))' = \Phi(c_{2,1} \tilde{a}_1 + c_{2,2} \tilde{a}_2)'$。

(A.51)に関して、積分部分がガウス分布の正規化定数になること、aなどが2次元だったことを考慮すると以下のようになる。
\begin{equation}
\begin{split}
C'(\lambda) = \frac{2|L_b|{c_1}^T c_2}{\pi^{2}} \int exp(-\frac{1}{2}\tilde{a}^T(I + 2c_2 {c_2}^T)\tilde{a})d\tilde{a}
= \frac{2|L_b|{c_1}^T c_2}{\pi^{2}} \sqrt{(2\pi)^2 |I + 2c_2 {c_2}^T|^{-1}}\\
= \frac{4|L_b|{c_1}^T c_2}{\pi|I + 2c_2 {c_2}^T|^{\frac{1}{2}}}
= \frac{4{c_1}^T c_2}{\pi|{L_b^T}^{-1}|^{\frac{1}{2}}|I + 2c_2 {c_2}^T|^{\frac{1}{2}}|{L_b}^{-1}|^{\frac{1}{2}}}
= \frac{4{b_1}^T L_b L_b^T b_2}{\pi|{L_b^T}^{-1}{L_b}^{-1} + 2{L_b^T}^{-1}L_b^T b_2 {b_2}^T L_b {L_b}^{-1}|^{\frac{1}{2}}}\\
= \frac{4{b_1}^T L_b L_b^T b_2}{\pi|(L_b L_b^{T})^{-1} + 2b_2 {b_2}^T|^{\frac{1}{2}}}
= \frac{4{b_1}^T L_b L_b^T b_2}{\pi|I + 2\lambda^2 b_1 b_1^T + 2b_2 {b_2}^T|^{\frac{1}{2}}}
= \frac{4{b_1}^T L_b L_b^T b_2}{\pi\Delta^{\frac{1}{2}}}
\end{split}
\end{equation}
ここで
\begin{equation}
\begin{split}
\Delta = |I + 2\lambda^2 b_1 b_1^T + 2b_2 {b_2}^T|
= 
\begin{vmatrix}
1 + 2\lambda^2 b_{1,1}^2 + 2 b_{2,1}^2 & 2\lambda^2 b_{1,1} b_{1, 2} + 2 b_{2, 1} b_{2, 2}\\
2\lambda^2 b_{1,1} b_{1, 2} + 2 b_{2, 1} b_{2, 2} & 1 + 2\lambda^2 b_{1,2}^2 + 2 b_{2,2}^2
\end{vmatrix}\\
= 1 + 2 \lambda (b_{1, 1}^2 + b_{1, 2}^2) + 2(b_{2, 1}^2 + b_{2, 2}^2) + 4\lambda^2 b_{2, 1}^2 b_{1, 2}^2 + 4\lambda^2 b_{1, 1}^2 b_{2, 2}^2 - 8\lambda^2 b_{1, 1} b_{1, 2} b_{2, 1} b_{2, 2}\\
= 1 + 2\lambda^2 b_1^T b_1 + 2 b_2^T b_2 + 4\lambda^2(b_1^T b_1 b_2^T b_2 - (b_1^T b_2)^2)
\end{split}
\end{equation}
となる。((A.55)は違っているのでは?)

シャーマン・モリソンの公式も確認しておく。
\begin{equation}
\begin{split}
I = (I + gg^T)(I + gg^T)^{-1} = (I + gg^T)(I - \frac{gg^T}{1 + g^T g})\\
= I + gg^T - \frac{gg^T}{1 + g^T g} - \frac{gg^Tgg^T}{1 + g^T g}
= I gg^T - \frac{gg^T + g(g^Tg)g^T}{1 + g^T g}
= I gg^T - \frac{(1 + g^Tg)gg^T}{1 + g^T g}\\
= I gg^T - gg^T = I
\end{split}
\end{equation}

この式を用いると(A.54)が言える。
\begin{equation}
\begin{split}
b_1^T L_b L_b^T b_2 =b_1^T (I + 2\lambda b_1 b_1^T)^{-1} b_2
= b_1^T (I - \frac{2 \lambda^2 b_1b_1^T}{1 + 2 \lambda^2 b_1^T b_1})b_2
=  b_1^T b_2 - \frac{2 \lambda^2 b_1^T b_1b_1^Tb_2}{1 + 2 \lambda^2 b_1^T b_1}\\
=  \frac{b_1^T b_2 + 2 \lambda^2 b_1^T b_1b_1^Tb_2 - 2 \lambda^2 b_1^T b_1b_1^Tb_2}{1 + 2 \lambda^2 b_1^T b_1}
=  \frac{b_1^T b_2}{1 + 2 \lambda^2 b_1^T b_1}
\end{split}
\end{equation}

これを踏まえると、
\begin{equation}
\begin{split}
C'(\lambda) = \frac{4{b_1}^T L_b L_b^T b_2}{\pi\Delta^{\frac{1}{2}}}
= \frac{4 \frac{b_1^T b_2}{1 + 2 \lambda^2 b_1^T b_1}}{\pi\Delta^{\frac{1}{2}}}
= \frac{4 b_1^T b_2}{\pi(1 + 2 \lambda^2 b_1^T b_1)\Delta^{\frac{1}{2}}}
\end{split}
\end{equation}
となり、(A.56)は成立する。

(A.57)のようにzを以下のように置く。
\begin{equation}
z = \frac{2\lambda b_1^T b_2}{\sqrt{(1 + 2\lambda^2 b_1^T b_1)(1 + 2b_2^T b_2)}}
\end{equation}
このとき、
\begin{equation}
\begin{split}
\frac{dz}{d\lambda} = \frac{2\lambda b_1^T b_2}{\sqrt{(1 + 2\lambda^2 b_1^T b_1)(1 + 2b_2^T b_2)}}
= (\frac{2 b_1^T b_2}{\sqrt{1 + 2b_2^T b_2}}\frac{\lambda}{\sqrt{1 + 2\lambda^2 b_1^T b_1}})'\\
= \frac{2 b_1^T b_2}{\sqrt{1 + 2b_2^T b_2}}(\frac{1}{\sqrt{1 + 2\lambda^2 b_1^T b_1}} - \frac{2\lambda b_1^T b_1}{\sqrt{1 + 2\lambda^2 b_1^T b_1}^{\frac{3}{2}}})
= \frac{2 b_1^T b_2}{\sqrt{1 + 2b_2^T b_2}}(\frac{1}{\sqrt{1 + 2\lambda^2 b_1^T b_1}^{\frac{3}{2}}})\\
= \frac{2 b_1^T b_2}{\sqrt{1 + 2b_2^T b_2}\sqrt{1 + 2\lambda^2 b_1^T b_1}^{\frac{3}{2}}}
\end{split}
\end{equation}

このとき、(A.58)が成り立つことを以下で確認する。
\begin{equation}
\begin{split}
C'(\lambda) = \frac{2}{\pi \sqrt{1 - z^2}}\frac{dz}{d\lambda}
= \frac{2}{\pi \sqrt{1 - \frac{4\lambda^2 (b_1^T b_2)^2}{(1 + 2b_2^T b_2)(1 + 2\lambda b_1^T b_1)}}}\frac{dz}{d\lambda}\\
= \frac{2\sqrt{(1 + 2b_2^T b_2)(1 + 2\lambda b_1^T b_1)}}{\pi \sqrt{1 + 2\lambda^2 b_1^T b_1 + 2b_2^T b_2 + 4\lambda^2(b_1^T b_1 b_2^T b_2 - (b_1^T b_2)^2)}}\frac{2 b_1^T b_2}{\sqrt{1 + 2b_2^T b_2}\sqrt{1 + 2\lambda^2 b_1^T b_1}^{\frac{3}{2}}}\\
= \frac{4 b_1^T b_2}{\pi \sqrt{1 + 2\lambda^2 b_1^T b_1 + 2b_2^T b_2 + 4\lambda^2(b_1^T b_1 b_2^T b_2 - (b_1^T b_2)^2)}(1 + 2\lambda^2 b_1^T b_1)}
= \frac{4 b_1^T b_2}{\pi \Delta^{\frac{1}{2}}(1 + 2\lambda^2 b_1^T b_1)}
\end{split}
\end{equation}

$y = arcsin z$の微分を考える。$z = sin y$なので、
\begin{equation}
\frac{dz}{dy} = cos z = \sqrt{1 - z^2}
\end{equation}
よって、
\begin{equation}
\frac{dy}{dz} = \frac{d}{dz} arcsin z = \frac{1}{\sqrt{1 - z^2}}
\end{equation}
$C'(\lambda)$を積分して、
\begin{equation}
C(\lambda) = \frac{2}{\pi} arcsin z + C
\end{equation}
しかし、先に示したように$C(0) = 0$なので、積分定数C = 0。よって、(A.59)が成り立つことがわかる。

これを考慮すると、以下のように(A.60)が成り立つことがわかる。
\begin{equation}
\begin{split}
C = C(1) = \frac{2}{\pi} arcsin (\frac{2 b_1^T b_2}{\sqrt{(1 + 2 b_1^T b_1)(1 + 2b_2^T b_2)}})
= \frac{2}{\pi} arcsin (\frac{2 e_1^T L L^T e_2}{\sqrt{(1 + 2 e_1^T L L^T e_1)(1 + 2e_2^T L L^T e_2)}})\\
= \frac{2}{\pi} arcsin (\frac{2 e_1^T \Sigma e_2}{\sqrt{(1 + 2 e_1^T \Sigma e_1)(1 + 2e_2^T \Sigma e_2)}})
= \frac{2}{\pi} arcsin (\frac{2 \Sigma_{1, 2}}{\sqrt{(1 + 2 \Sigma_{1, 1})(1 + 2\Sigma_{2, 2})}})
\end{split}
\end{equation}
\end{document}
