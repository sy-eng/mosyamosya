\documentclass{jsarticle}
\usepackage{amsmath,amssymb,amsfonts}
\begin{document}
6.1.1.2の式展開を見直す。

そもそも、(6.4)のようにKLダイバージェンスを最小化する。qに関して、
(6.4)のように、$q(Z,W)$を考慮するが、それらはXを引数としても良いし、
その関数は$\psi, \xi$をパラメータとしている。よって、(6.9)の式を最小化する。

(3.10)の定義式に(6.5)を代入して、
\begin{equation}
\begin{split}
D_{KL}[q(Z, W ; X, \psi, \xi) || p(Z, W | X)] 
= - \int \int q(Z, W ; X, \psi, \xi) \, ln \, \frac{p(Z, W | X)}{q(Z, W ; X, \psi, \xi)} dZdW\\
= - \int \int q(Z, W ; X, \psi, \xi) \, ln \, p(Z, W | X) dZdW + \int \int q(Z, W ; X, \psi, \xi) \, ln \, q(Z, W ; X, \psi, \xi) dZdW \\
= - \int \int q(Z, W ; X, \psi, \xi) \, ln \, \frac{p(Z, W, X)}{p(X)} dZdW + \int \int q(Z, W ; X, \psi, \xi) \, ln \, q(Z,; X, \psi)q(W ; \xi) dZdW\\
= - \int \int q(Z, W ; X, \psi, \xi) \, ln \, p(Z, W, X) dZdW + \int \int q(Z, W ; X, \psi, \xi) \, ln \, p(X) dZdW \\ + \int \int q(Z, W ; X, \psi, \xi) \, ln \, q(Z,; X, \psi) dZdW + \int \int q(Z, W ; X, \psi, \xi) \, ln \, q(W ; \xi) dZdW \\
= - \mathbb{E}_{q(Z, W ; X, \psi, \xi)} [ln \, p(Z, W, X) ] + \mathbb{E}_{q(Z, W ; X, \psi, \xi)}[ln \, p(X)] + \mathbb{E}_{q(Z, W ; X, \psi, \xi)} [ln \, q(Z,; X, \psi)] + \mathbb{E}_{q(Z, W ; X, \psi, \xi)} [ln \, q(W ; \xi)]\\
= - (\mathbb{E}_{q(Z, W ; X, \psi, \xi)} [ln \, p(Z, W, X) ] - \mathbb{E}_{q(Z, W ; X, \psi, \xi)} [ln \, q(Z,; X, \psi)] - \mathbb{E}_{q(Z, W ; X, \psi, \xi)} [ln \, q(W ; \xi)]) + \mathbb{E}_{q(Z, W ; X, \psi, \xi)}[ln \, p(X)] \\
\end{split}
\end{equation}
この様になり、(6.9)が求まった。この式が求まれば、(6.10)は自明。

(6.10)は(6.3),(6,7)を考慮して、以下のようになる。
\begin{equation}
\begin{split}
\mathcal{L}(\psi, \xi) = \mathbb{E}_{q(Z, W ; X, \psi, \xi)} [ln \, p(Z, W, X) ] - \mathbb{E}_{q(Z, W ; X, \psi, \xi)} [ln \, q(Z; X, \psi)] - \mathbb{E}_{q(Z, W ; X, \psi, \xi)} [ln \, q(W ; \xi)]\\
= \mathbb{E}_{q(Z, W ; X, \psi, \xi)} [ln \, p(W)\prod_n [p(x_n | z_n, W) p(z_n)] ] - \mathbb{E}_{q(Z, W ; X, \psi, \xi)} [ln \, \prod_n q(z_n; x_n, \psi)] - \mathbb{E}_{q(Z, W ; X, \psi, \xi)} [ln \, q(W ; \xi)]\\
= \mathbb{E}_{q(Z, W ; X, \psi, \xi)} [ln \, p(W)] + \sum_n \mathbb{E}_{q(Z, W ; X, \psi, \xi)} [ln \, p(x_n | z_n, W)] + \sum_n \mathbb{E}_{q(Z, W ; X, \psi, \xi)} [ln \, p(z_n)]\\ - \sum_n \mathbb{E}_{q(Z, W ; X, \psi, \xi)} [ln \, q(z_n; x_n, \psi)] - \mathbb{E}_{q(Z, W ; X, \psi, \xi)} [ln \, q(W ; \xi)]\\
= \sum_n (\mathbb{E}_{q(Z, W ; X, \psi, \xi)} [ln \, p(x_n | z_n, W)] + \mathbb{E}_{q(Z, W ; X, \psi, \xi)} [ln \, p(z_n)] - \mathbb{E}_{q(Z, W ; X, \psi, \xi)} [ln \, q(z_n; x_n, \psi)])\\ + \mathbb{E}_{q(Z, W ; X, \psi, \xi)} [ln \, p(W)] - \mathbb{E}_{q(Z, W ; X, \psi, \xi)} [ln \, q(W ; \xi)]\\
\end{split}
\end{equation}
これをもとに、ミニバッチのELBOを考えると(6.11)のようになる。
\begin{equation}
\begin{split}
\mathcal{L}(\psi, \xi) = \frac{N}{M}\sum_{n \in \mathcal{S}} (\mathbb{E}_{q(Z, W ; X, \psi, \xi)} [ln \, p(x_n | z_n, W)] + \mathbb{E}_{q(Z, W ; X, \psi, \xi)} [ln \, p(z_n)] - \mathbb{E}_{q(Z, W ; X, \psi, \xi)} [ln \, q(z_n; x_n, \psi)])\\ + \mathbb{E}_{q(Z, W ; X, \psi, \xi)} [ln \, p(W)] - \mathbb{E}_{q(Z, W ; X, \psi, \xi)} [ln \, q(W ; \xi)]\\
\end{split}
\end{equation}
ここで、以下の勾配を考える。
\begin{equation}
\begin{split}
\nabla_{\xi} (\mathbb{E}_{q(Z, W ; X, \psi, \xi)} [ln \, p(z_n)])
= \nabla_{\xi} (\int \int q(Z, W ; X, \psi, \xi) \, ln \, p(z_n) dZdW)
= \nabla_{\xi} (\int \int q(Z; X, \psi) q(W, \xi) \, ln \, p(z_n) dZdW)\\
= \nabla_{\xi} (\int q(W, \xi) dW \int q(Z; X, \psi)  \, ln \, p(z_n) dZ)
= \nabla_{\xi} (\int q(Z; X, \psi)  \, ln \, p(z_n) dZ) = 0
\end{split}
\end{equation}
\begin{equation}
\begin{split}
\nabla_{\xi} (\mathbb{E}_{q(Z, W ; X, \psi, \xi)} [ln \, q(z_n; x_n, \psi)])
= \nabla_{\xi} (\int \int q(Z, W ; X, \psi, \xi) \, ln \, q(z_n; x_n, \psi) dZdW)\\
= \nabla_{\xi} (\int \int q(Z; X, \psi) q(W, \xi) \, ln \, q(z_n; x_n, \psi) dZdW)
= \nabla_{\xi} (\int q(W, \xi) dW \int q(Z; X, \psi)  \, ln \, q(z_n; x_n, \psi) dZ)\\
= \nabla_{\xi} (\int q(Z; X, \psi)  \, ln \, q(z_n; x_n, \psi) dZ) = 0
\end{split}
\end{equation}
\begin{equation}
\begin{split}
\nabla_{\psi} (\mathbb{E}_{q(Z, W ; X, \psi, \xi)} [ln \, p(W)])
= \nabla_{\psi} (\int \int q(Z, W ; X, \psi, \xi) \, ln \, p(W) dZdW)
= \nabla_{\psi} (\int \int q(Z; X, \psi) q(W, \xi) \, ln \, p(W) dZdW)\\
= \nabla_{\psi} (\int q(Z; X, \psi) dZ \int q(W, \xi)  \, ln \, p(W) dW)
= \nabla_{\psi} (\int q(W, \xi)  \, ln \, p(W) dW) = 0
\end{split}
\end{equation}
\begin{equation}
\begin{split}
\nabla_{\psi} (\mathbb{E}_{q(Z, W ; X, \psi, \xi)} [ln \, q(W; \xi)])
= \nabla_{\psi} (\int \int q(Z, W ; X, \psi, \xi) \, ln \, q(W; \xi) dZdW)\\
= \nabla_{\psi} (\int \int q(Z; X, \psi) q(W, \xi) \, ln \, q(W; \xi) dZdW)
= \nabla_{\psi} (\int q(Z; X, \psi) dZ \int q(W, \xi)  \, ln \, q(z_n; x_n,q(W; \xi) dW)\\
= \nabla_{\psi} (\int q(W, \xi)  \, ln \, q(z_n; x_n,q(W; \xi) dW) = 0
\end{split}
\end{equation}
これらを踏まえると、(6.12),(6.13)が求まる。なお、上記と同様にすると(6.12),(6.13)の残った項は微分しても微分する変数が関数の中に残るため、すぐに0になるとは言えないことがわかる。
\end{document}
