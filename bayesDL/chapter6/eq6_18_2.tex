\documentclass{jsarticle}
\usepackage{amsmath,amssymb,amsfonts}
\begin{document}
[59]をもとに(6.18)について再度考える。

これは$p(Y_U, Z_A, Z_U| X_A, X_U, Y_A)$ではなく、$p(Y_A, Y_U, Z_A, Z_U| X_A, X_U)$を検討することになると考えられる。

(6.14)を考慮すると以下のようになる。
\begin{equation}
\begin{split}
\label{p}
p(Y_U, Z_A, Z_U | Y_A, X_A, X_U)
= \frac{p(Y_A, Y_U, Z_A, Z_U, X_A, X_U)}{p(Y_A, X_A, X_U)}
= \frac{p(X_A | Y_A, Z_A)p(Y_A)p(Z_A)p(X_U| Y_U, Z_U)p(Y_U)p(Z_U)}{p(Y_A, X_A, X_U)} \\
= p(X_A | Y_A, Z_A)p(Z_A)p(X_U| Y_U, Z_U)p(Y_U)p(Z_U)\frac{p(Y_A)}{p(Y_A, X_A, X_U)} 
= p(X_A | Y_A, Z_A)p(Z_A)p(X_U| Y_U, Z_U)p(Y_U)p(Z_U)exp\{c\}
\end{split}
\end{equation}
3つめの等号のあとの分数の部分がデータとしてわかっているので定数。

また、qを平均場近似を用いて、以下のようにする。
\begin{equation}
\begin{split}
\label{q1}
q \equiv q(Y_U, Z_A, Z_U, \psi_2; X_A, X_U, Y_A, \psi_{1a}, \psi_{1b}, \psi_3) 
\approx q(Z_A; X_A, X_U, Y_A, \psi_{1a}, \psi_{1b}, \psi_3)q(Z_U ; X_A, X_U, Y_A, \psi_{1a}, \psi_{1b}, \psi_3)\\
q(Y_A ; X_A, X_U, Y_A, \psi_{1a}, \psi_{1b}, \psi_2, \psi_3)q(Y_U, \psi_2 ; X_A, X_U, Y_A, \psi_{1a}, \psi_{1b}, \psi_3)\\
\approx q(Z_A; X_A, Y_A, \psi_{1a})q(Z_U ; X_U, \psi_{1b})q(Y_U ; X_U, \psi_2)q(Y_A ; X_A, \psi_2)q(\psi_2 ; \psi_3)
\end{split}
\end{equation}
本と異なり、$\psi_2$がパラメータ$psi_3$によって決まる確率変数としている。

これらは(6.6),(6.15)-(6.17)のように定数や、得られたデータを用いて、表される。
また、対称ディリクレ分布を仮定して、$q(\psi_2 | \psi_3=(\alpha)) = Dir(\psi_2 |( \alpha)) = C_D((\alpha))\prod_k \psi_{2k}^{\alpha - 1}$

(3.10)のKLダイバージェンスの定義を考慮すると、
\begin{equation}
\begin{split}
\label{kl}
D_{KL}(q(Y_U, Z_A, Z_U, \psi_2 ; X_A, Y_A, X_U, \psi_{1a}, \psi_{1b}, \psi_3) || p(Y_U, Z_A, Z_U | X_A, X_U, Y_A))
= - \mathbb{E}_{q(Z_A; X_A, Y_A, \psi_{1a})} [ln \, p(X_A | Y_A, Z_A)]\\
- \mathbb{E}_{q(Z_A; X_A, Y_A, \psi_{1a})} [ln \, p(Z_A)]
+ \mathbb{E}_{q(Z_A; X_A, Y_A, \psi_{1a})} [ln \, q(Z_A; X_A, Y_A, \psi_{1a})] \\
- \mathbb{E}_{q(Y_U; X_U, \psi_2)q(\psi_2 ; \psi_3)q(Z_U; X_U, \psi_{1b})}[ln \, p(X_U| Y_U, Z_U)]\\
- \mathbb{E}_{q(Y_U; X_U, \psi_2)q(\psi_2 ; \psi_3)} [ln \, p(Y_U)] - \mathbb{E}_{q(Z_U; X_U, \psi_{1b})} [ln \, p(Z_U)]\\
+ \mathbb{E}_{q(Y_U; X_U, \psi_2)q(\psi_2 ; \psi_3)} [ln \, q(Y_U; X_U, \psi_2)]
+ \mathbb{E}_{q(Z_U; X_U, \psi_{1b})} [ln \, q(Z_U; X_U, \psi_{1b})]\\
+ \mathbb{E}_{q(Y_A; X_A, \psi_2)q(\psi_2 ; \psi_3)} [ln \, q(Y_A; X_A, \psi_2)] 
+ \mathbb{E}_{q(\psi_2 ; \psi_3)} [ln \, q(\psi_2 ; \psi_3)] + c
\end{split}
\end{equation}
これで(6.18)に相当する式が求まった。

(6.18)との違いは、期待値を取る確率分布qの若干の違いもあるが、本質的には以下のみ異なる。
\begin{equation}
\begin{split}
\mathbb{E}_{q(Y_A; X_A, \psi_2)q(\psi_2 ; \psi_3)} [ln \, q(Y_A; X_A, \psi_2)] + \mathbb{E}_{q(\psi_2 ; \psi_3)} [ln \, q(\psi_2 ; \psi_3)] = \\
\mathbb{E}_{q(Y_A; X_A, \psi_2)q(\psi_2 ; \psi_3)} [ln \, q(Y_A; X_A, \psi_2)] + \mathbb{E}_q(Y_A; X_A, \psi_2){q(\psi_2 ; \psi_3)} [ln \, q(\psi_2 ; \psi_3)]\\
= \mathbb{E}_{q(Y_A; X_A, \psi_2)q(\psi_2 ; \psi_3)} [ln \, q(Y_A; X_A, \psi_2) + ln \, q(\psi_2 ; \psi_3)]
\end{split}
\end{equation}
$q(\psi_2 ; \psi_3)$に対称ディリクレ分布を仮定しているので、$\psi_2$の各要素は、ほぼ同一だと想定され、$q(Y_A; X_A, \psi_2)$は以下のように仮定する。
\begin{equation}
\begin{split}
q(Y_A; X_A, \psi_2) \equiv q(Y_A; \psi_2) = Cat(Y_A | \psi_2) = \prod_a \prod_k \psi_{2k}^{y_{ak}}
\end{split}
\end{equation}
そうした場合、
\begin{equation}
\begin{split}
ln \, q(Y_A; X_A, \psi_2) + ln \, q(\psi_2 ; \psi_3)  = (\sum_k (\sum_a y_{ak} + \alpha - 1) ln \, \psi_{2k}) + c
\end{split}
\end{equation}
$\psi_2$の要素がほぼ同一だと$\sum_a y_{ak}$が、kによらず、ほぼ同一になり、$(\sum_a y_{ak} + \alpha - 1)$が定数とみなせる。
よって、
\begin{equation}
\begin{split}
ln \, q(Y_A; X_A, \psi_2) + ln \, q(\psi_2 ; \psi_3)  = (\sum_k (\sum_a y_{ak} + \alpha - 1) ln \, \psi_{2k}) + c = \beta ln \, \psi_{2}
\end{split}
\end{equation}
よって、(符号は違う気がするが、)(6.22)のような式に妥当性があるように見える。
\end{document}
