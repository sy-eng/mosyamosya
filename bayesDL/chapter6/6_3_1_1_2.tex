\documentclass{jsarticle}
\usepackage{amsmath,amssymb,amsfonts}
\usepackage[dvipdfmx]{hyperref}
\usepackage{pxjahyper}
\begin{document}
本のP.173にある、IBPについて考える。

\href{https://cocosci.princeton.edu/tom/papers/ibptr.pdf}{論文}の(15)(16)を見ると、CRPに関してはわかりやすい。

(16)を考慮して、
\begin{equation}
P(c) = \prod_n^N P(c_n | c_1, \cdots, c_{n-1}) = \frac{1}{(N-1-\alpha)!} \alpha^{K_+} \prod_k^{K_+} (m_k-1)!
\end{equation}
となる。ここで、otherwiseの方は0でないクラスの数なので、$K_{+}$回選択される。それ以外のときは、(16)で$m_k$がそれぞれカウントアップされるので、階乗の項が出てくる。(しかし、追加される前の個数をかけるので、-1がつく。)

逆に(15)を考慮すると、
\begin{equation}
P(c) = \alpha^{K_+}(\prod_k^{K_+} (m_k-1)!)\frac{\Gamma(\alpha)}{\Gamma(N+\alpha)}
= \alpha^{K_+}(\prod_k^{K_+} (m_k-1)!)\frac{\Gamma(\alpha)}{(N-1+\alpha)!\Gamma(\alpha)}
= \alpha^{K_+}(\prod_k^{K_+} (m_k-1)!)\frac{1}{(N-1+\alpha)!}
\end{equation}
よって、(16)の手続きで、(15)の確率分布が出てくる。

なお、$c_i$はすでに選ばれているk個の料理と新しい料理k+1のいずれかになる。
確率として、$c_1, \cdots, c_{i-1}$に条件付けられると、$\sum_k P(c_i | c_1, \cdots, c_{i-1})=1$となる。
i番目のサンプリング前で、$\sum m_k = i - 1$になることを考慮すると、
\begin{equation}
P(c_i | c_1, \cdots, c_{i-1}) = \sum_k \frac{m_k}{i - 1 + \alpha} + \frac{\alpha}{i - 1 + \alpha}
= \frac{i - 1}{i - 1 + \alpha} + \frac{\alpha}{i - 1 + \alpha} = 1
\end{equation}

(6.51)に関して考える。
\begin{equation}
\begin{split}
p([M]) = \frac{(\alpha\beta)^{H_+}}{\prod_{i \geq 1}H_i!} exp(-(\alpha \sum_{j = 1}^{N}\frac{\beta}{j + \beta - 1})) \prod_{h=1}^{H_+}  \frac{\Gamma(N_h)\Gamma(N - N_h + \beta)}{\Gamma(N + \beta) }\\
= \frac{(\alpha\beta)^{H_+}}{\prod_{i \geq 1}H_i!} exp(-(\alpha \sum_{j = 1}^{N}\frac{\beta}{j + \beta - 1})) \prod_{h=1}^{H_+}  \frac{(N_h - 1)! (N - N_h + \beta - 1)! \Gamma(\beta)}{(N + \beta - 1)!\Gamma(\beta) }\\
= \frac{(\alpha\beta)^{H_+}}{\prod_{i \geq 1}H_i!} exp(-(\alpha \sum_{j = 1}^{N}\frac{\beta}{j + \beta - 1})) \prod_{h=1}^{H_+}  \frac{(N_h - 1)! (N - N_h + \beta - 1)!}{(N + \beta - 1)! }
\end{split}
\end{equation}
なお、
\begin{equation}
H_{+} = \sum_{i \geq 1} H_i
\end{equation}
IBPの手続きは、P.173の1,2のように行う。
ただ、n=1のときに$\frac{\alpha\beta}{n+\beta-1}=\alpha$なので、記載方法を変えると、
\begin{equation}
\begin{split}
p(z_{nh}=1) = \frac{N_{nh}}{n + \beta - 1}(すでに選択されている料理hが選ばれる確率。\\
nはこのときまでに選んだ人の数,N_{nh}はその時までにhを選んでいる人の数。)\\
p_n(x_n | \frac{\alpha\beta}{n + \beta -1}) = Poi(x_n | \frac{\alpha\beta}{n + \beta -1}) = \frac{{\frac{\alpha\beta}{n + \beta -1}}^{x_n}}{x_n !}e^{-\frac{\alpha\beta}{n + \beta -1}}\\
(n番目の人がこれまで取られていない新しいとる料理の数)。
\end{split}
\end{equation}
このとき、
\begin{equation}
H_{+} = \sum_n^N x_n
\end{equation}
\begin{equation}
p_n(z_{nh}=0) = \frac{n + \beta - 1 - N_{nh}}{n + \beta - 1}
\end{equation}
あるhに関して、j番目で初めて、選択されたとする。このとき、hに関して、
\begin{equation}
p(h) = \prod_{z_{nh} = 1,n>j}^{N} N_{nh} \prod_{z_{nh} \neq 1, n>j}(n + \beta - 1 - N_{nh})\prod_{n = j+1}^{N} \frac{1}{n + \beta - 1}
\end{equation}


\end{document}
