\documentclass{jsarticle}
\usepackage{amsmath,amssymb,amsfonts}
\usepackage[dvipdfmx]{hyperref}
\usepackage{pxjahyper}
\begin{document}
6.3.1.1の式変形を考える。式変形は[42]に詳しいとあるのでそれも参考に考える。

$m_{n, h} \in \{0, 1\}$がベルヌーイ分布(3.20)から生成されるので、
\begin{equation}
p(m_{n, h} | \pi_h) = \pi_h^{m_{n, h}}(1 - \pi_h)^{1 - m_{n, h}}
\end{equation}
また、$\pi_h$はベータ分布(3.49)から生成されるので
\begin{equation}
p(\pi_h) = \frac{\Gamma(\frac{\alpha\beta}{H} + \beta)}{\Gamma(\frac{\alpha\beta}{H})\Gamma(\beta)}\pi_h^{\frac{\alpha\beta}{H} - 1}(1 - \pi_h)^{\beta - 1}
\end{equation}
なお、[42]では$H = H$,$\beta = 1$となっている。
\begin{equation}
\begin{split}
p(M) = \prod_{h=1}^H \prod_{n=1}^N p(m_{n, h})  = \prod_{h=1}^H \int p(\pi_h) \{ \prod_{n=1}^N p(m_{n, h} | \pi_h)\} d \pi_h\\
= \prod_{h=1}^H \int \frac{\Gamma(\frac{\alpha\beta}{H} + \beta)}{\Gamma(\frac{\alpha\beta}{H})\Gamma(\beta)}\pi_h^{\frac{\alpha\beta}{H} - 1}(1 - \pi_h)^{\beta - 1} \prod_{n=1}^N \pi_h^{m_{n, h}}(1 - \pi_h)^{1 - m_{n, h}} d \pi_h\\
= \prod_{h=1}^H \frac{\Gamma(\frac{\alpha\beta}{H} + \beta)}{\Gamma(\frac{\alpha\beta}{H})\Gamma(\beta)} \int \pi_h^{\frac{\alpha\beta}{H} - 1 + \sum_{n = 1}^N m_{n, h}}(1 - \pi_h)^{\beta - 1 + N - \sum_{n = 1}^N m_{n, h}}d \pi_h\\
= \prod_{h=1}^H \frac{\Gamma(\frac{\alpha\beta}{H} + \beta)}{\Gamma(\frac{\alpha\beta}{H})\Gamma(\beta)} \int \pi_h^{\frac{\alpha\beta}{H} - 1 + N_h}(1 - \pi_h)^{\beta - 1 + N - N_h}d \pi_h
= \prod_{h=1}^H \frac{\Gamma(\frac{\alpha\beta}{H} + \beta)}{\Gamma(\frac{\alpha\beta}{H})\Gamma(\beta)}\frac{\Gamma(\frac{\alpha\beta}{H} + N_h)\Gamma(N - N_h + \beta)}{\Gamma(\frac{\alpha\beta}{H} + \beta + N)}\\
\end{split}
\end{equation}
最後の等式については、(3.49)のベータ分布の式を見ると、積分部分が、正規化定数の逆数になっていることがわかる。そのため、
それを踏まえて、$Beta(\frac{\alpha\beta}{H} + N_h, \beta + N - N_h)$の正規化定数の逆数を考えれば良い。
この式は[42]の(2)と同等になる。一見同等ではないが、(3.24)から$\Gamma(1) = 1$、(3.25)から$\Gamma(x) = (x - 1)\Gamma(x - 1)$を考慮して、$\beta = 1, Z = M, H = H, z_{ik} = m_{n, h}, N_h = \sum_{n=1}^N z_{ik} = \sum_{n=1}^N m_{n, h} = N_h$なので、
\begin{equation}
\begin{split}
p(Z) = p(M) = \prod_{h=1}^H \frac{\Gamma(\frac{\alpha\beta}{H} + \beta)}{\Gamma(\frac{\alpha\beta}{H})\Gamma(\beta)}\frac{\Gamma(\frac{\alpha\beta}{H} + N_h)\Gamma(N - N_h + \beta)}{\Gamma(\frac{\alpha\beta}{H} + \beta + N)}
= \prod_{h=1}^H \frac{\Gamma(\frac{\alpha}{H} + 1)}{\Gamma(\frac{\alpha}{H})\Gamma(1)}\frac{\Gamma(\frac{\alpha}{H} + N_h)\Gamma(N - N_h + 1)}{\Gamma(\frac{\alpha}{H} + 1 + N)}\\
= \prod_{h=1}^H \frac{\frac{\alpha}{H}\Gamma(\frac{\alpha}{H})}{\Gamma(\frac{\alpha}{H})}\frac{\Gamma(\frac{\alpha}{H} + N_h)\Gamma(N - N_h + 1)}{\Gamma(\frac{\alpha}{H} + 1 + N)}
= \prod_{k=1}^H \frac{\frac{\alpha}{H}\Gamma(N_h + \frac{\alpha}{H})\Gamma(N - N_h + 1)}{\Gamma(N + 1 + \frac{\alpha}{H})}\\
\end{split}
\end{equation}
となり、等しいことがわかる。

(6.51)にあるように、p([M])を考える。Hで並び替えするとH!になるが、同じバイナリ列iが$H_i$個あると、入れ替えたものがおなじになるので個数が$1 / H_i!$になる。また、バイナリ列はN行あるが、それぞれ\{1, 0\}なので、$2^N$種類ありうる。$0! = 1$も考慮すると、[m]は
\begin{equation}
\frac{H!}{\prod_{i=0}^{2^N - 1}H_i!}
\end{equation}
種類ある。(この式は[42]の2.2の最後の行に記載がある。本ではiは1からになっているが、種類を表すものなのでなんでも良い。種類は[42]にあるように最大$2^N$種類ある。)

それぞれ、同じ確率なので、(6.51)の2行目まで、[42]の(3)の式のようになる。
\begin{equation}
p([M]) = \frac{H!}{\prod_{i=0}^{2^N - 1}H_i!} \prod_{h=1}^H \frac{\Gamma(\frac{\alpha\beta}{H} + \beta)\Gamma(N_h + \frac{\alpha\beta}{H})\Gamma(N - N_h + \beta)}{\Gamma(\frac{\alpha\beta}{H})\Gamma(\beta)\Gamma(\frac{\alpha\beta}{H} + \beta + N)}
\end{equation}

iは自由としているが、$N_h = 0$となるのはすべての列が0になる、1種類なので、すべて0となる列の個数を$H_0$とし、
$H = H_0 + H_+ = \sum_{i=0}^{2^N - 1} H_i = H_0 + \sum_{i=1}^{2^N - 1} H_i$で、$H_+ = \sum_{i=1}^{2^N - 1} H_i$とする。

[42]の(3)について、$N_h = 0$となる列を分けて考える。上記のガンマ関数の特性より$\Gamma(n+1) = n!, n \in N$にも注意し、$\beta = 1$とすると、
\begin{equation}
\begin{split}
p([M]) = \frac{H!}{\prod_{i=0}^{2^N - 1}H_i!} \prod_{h=1}^H \frac{\frac{\alpha}{H}\Gamma(N_h + \frac{\alpha}{H})\Gamma(N - N_h + 1)}{\Gamma(N + 1 + \frac{\alpha}{H})}\\
= \frac{H!}{H_0! \prod_{i=1}^{2^N - 1}H_i!} (\frac{\frac{\alpha}{H}}{\Gamma(N + 1 + \frac{\alpha}{H})})^H (\Gamma(\frac{\alpha}{H})\Gamma(N + 1))^{H_0} \prod_{h=1, N_h \neq 0}^{H_+} \Gamma(N_h + \frac{\alpha}{H})\Gamma(N - N_h + 1)\\
= \frac{H!}{H_0! \prod_{i=1}^{2^N - 1}H_i!} (\frac{\frac{\alpha}{H}}{\prod_{j = 0}^N (j + \frac{\alpha}{H})\Gamma(\frac{\alpha}{H})})^H (\Gamma(\frac{\alpha}{H})N!)^{H - H_+} \prod_{h=1, N_h \neq 0}^{H_+} (\prod_{j=0}^{N_h - 1}(j + \frac{\alpha}{H}))\Gamma(\frac{\alpha}{H})(N - N_h)!\\
= \frac{H!}{H_0! \prod_{i=1}^{2^N - 1}H_i!} (\frac{1}{\prod_{j = 1}^N (j + \frac{\alpha}{H})})^H N!^{H} (\frac{\alpha}{H})^{H_+}\prod_{h=1, N_h \neq 0}^{H_+} \frac{\prod_{j=1}^{N_h - 1}(j + \frac{\alpha}{H})(N - N_h)!}{N!}\\
= \frac{\alpha^{H_+}}{\prod_{i=1}^{2^N - 1}H_i!} \frac{H!}{H_0!H^{H_+}}(\frac{N!}{\prod_{j = 1}^N (j + \frac{\alpha}{H})})^H \prod_{h=1, N_h \neq 0}^{H_+} \frac{(N - N_h)!\prod_{j=1}^{N_j - 1}(j + \frac{\alpha}{H})}{N!}\\
\end{split}
\end{equation}
このさきの詳細は元論文でも別の資料(Infinite latent feature models and the Indian buffet process. Technical Report 2005-001)を参照するようになっている。別資料を参照すると、そもそも、行列はスパースであることが仮定されている。
つまり、
\begin{equation}
H_+ << H, H_0 \approx H
\end{equation}
そのため、$H_+$が有限であること、$N_h$が有限であることが仮定されている。(これを考えると、行列の図はスパースでない。と思ったが、よく考えると、本当は無限の列を考えるので、結果有限になり、残りの無限が0になり、全体として考えるとスパース。サンプリングされる料理の数は有限なので、結果的にスパースになっている。)

別資料のAppendixにあるように、それぞれの項について、検討する。(60)-(62)に関して考えると、
\begin{equation}
\begin{split}
\frac{H!}{H_0!H^{H_+}} = \frac{\prod_{h=1}^{H_+}(H - h + 1)}{H^{H_+}} =  \frac{\prod_{h=1}^{H_+}(H - (h - 1))}{H^{H_+}}=\\
\frac{H^{H_+} - H^{H_+ - 1}\sum_{h = 0}^{H_+ - 1} h + H^{H_+ - 2}(\sum_{h = 0}^{H_+ - 1}\sum_{j = h + 1}^{H_+ - 1} hj) + ... + (-1)^{H_+ - 2}(\sum_{h=1}^{H_+ - 1}\frac{1}{h})(H_+ - 1)! + (-1)^{H_+ - 1}(H_+ - 1)!}{H^{H_+}}\\
\leq 1 + \frac{{H_+}^2}{H} + \frac{{H_+}^3}{H^2} + ... + \frac{{H_+}^{H_+ - 1}}{H^{H_+ - 2}} + \frac{{H_+}^{H_+}}{H^{H_+ - 1}} \leq 1 + \frac{{H_+}^{H_+ + 1}}{H}
\end{split}
\end{equation}
$H_+$が有限で、Hを無限への極限とするので、
\begin{equation}
\begin{split}
\lim_{H \rightarrow \infty} \frac{H!}{H_0!H^{H_+}} = \lim_{H \rightarrow \infty}(1 + \frac{{H_+}^{H_+ + 1}}{H}) = 1
\end{split}
\end{equation}
なお、各項を-としたものに関しても考えると、それよりは大きくなる。しかし、これも極限で1になるので、1になることがわかる。

次に(63)を考える。これも、上記と同様に考えられる。
\begin{equation}
\begin{split}
(N_h - 1)! \leq \prod_{j=1}^{N_h - 1}(j + \frac{\alpha}{H}) = \\
(N_h - 1)! + \frac{\alpha}{H}(N_h - 1)!\sum_{j=1}^{N_h - 1}\frac{1}{j} + (\frac{\alpha}{H})^2 (N_h - 1)!\sum_{j=1}^{N_h - 2}\sum_{k=j + 1}^{N_h - 1}\frac{1}{jk} + ... + (\frac{\alpha}{H})^{N_h - 2} \sum_{j=1}^{N_h - 1} j + (\frac{\alpha}{H})^{N_h - 1}\\
= (N_h - 1)! + \frac{\alpha^{N_h - 1}}{H}(N_h - 1)^{N_h}
\end{split}
\end{equation}
$N_h, \alpha$が有限なので、
\begin{equation}
\lim_{H \rightarrow \infty} \prod_{j=1}^{N_h - 1}(j + \frac{\alpha}{H}) = (N_h - 1)!
\end{equation}

これらを踏まえると、
\begin{equation}
\begin{split}
p([M]) = \frac{\alpha^{H_+}}{\prod_{i=1}^{2^N - 1}H_i!} \frac{H!}{H_0!H^{H_+}}(\frac{N!}{\prod_{j = 1}^N (j + \frac{\alpha}{H})})^H \prod_{h=1, N_h \neq 0}^{H_+} \frac{(N - N_h)!\prod_{j=1}^{N_h - 1}(j + \frac{\alpha}{H})}{N!}\\
\approx \frac{\alpha^{H_+}}{\prod_{i=1}^{2^N - 1}H_i!} (\frac{N!}{\prod_{j = 1}^N (j + \frac{\alpha}{H})})^H \prod_{h=1, N_h \neq 0}^{H_+} \frac{(N - N_h)!(N_h - 1)!}{N!}\\
\end{split}
\end{equation}

最後に(64)-(70)を考える。(64)-(66)に関しては記載の通りでわかりやすい。

(67)に関して、$\lim_{H\rightarrow\infty}(1+\frac{x}{K})^H$を考える。これが収束すれば、以下が成り立つ。
\begin{equation}
\lim_{H\rightarrow\infty}(\frac{1}{1+\frac{x}{K}})^H = \frac{1}{\lim_{H\rightarrow\infty}(1+\frac{x}{K})^H}
\end{equation}
\href{https://ja.wikipedia.org/wiki/%E3%83%8D%E3%82%A4%E3%83%94%E3%82%A2%E6%95%B0}{ネイピア数の定義}を考え、
変数変換も考えるとこのリンクにあるように以下の性質が成り立つ。
\begin{equation}
\lim_{H\rightarrow\infty}(1+\frac{x}{K})^H = \lim_{\frac{H}{x}\rightarrow\infty}((1+\frac{x}{K})^\frac{H}{x})^x = e^{x}
\end{equation}
よって、(67)が成り立つ。これを使うと、(68)が成り立ち、(70)が導かれる。

これらから、元論文の(4)が導かれる。
\begin{equation}
\begin{split}
p([M]) = \frac{\alpha^{H_+}}{\prod_{i=1}^{2^N - 1}H_i!} \frac{H!}{H_0!H^{H_+}}(\frac{N!}{\prod_{j = 1}^N (j + \frac{\alpha}{H})})^H \prod_{h=1, N_h \neq 0}^{H_+} \frac{(N - N_h)!\prod_{j=1}^{N_j - 1}(j + \frac{\alpha}{H})}{N!}\\
\approx \frac{\alpha^{H_+}}{\prod_{i=1}^{2^N - 1}H_i!} e^{-\alpha H_N} \prod_{h=1, N_h \neq 0}^{H_+} \frac{(N - N_h)!(N_h - 1)!}{N!}\\
\end{split}
\end{equation}

さて、上記では$\beta = 1$としていたが、$\beta$をそのままとして、再度検討する。

\begin{equation}
\begin{split}
p([M]) = \frac{H!}{\prod_{i=0}^{2^N - 1}H_i!} \prod_{h=1}^H \frac{\Gamma(\frac{\alpha\beta}{H} + \beta)\Gamma(N_h + \frac{\alpha\beta}{H})\Gamma(N - N_h + \beta)}{\Gamma(\frac{\alpha\beta}{H})\Gamma(\beta)\Gamma(\frac{\alpha\beta}{H} + \beta + N)}\\
= \frac{H!}{H_0! \prod_{i=1}^{2^N - 1}H_i!} (\frac{\Gamma(\frac{\alpha\beta}{H} + \beta)\Gamma(\frac{\alpha\beta}{H})\Gamma(N + \beta)}{\Gamma(\frac{\alpha\beta}{H})\Gamma(\beta)\Gamma(\frac{\alpha\beta}{H} + \beta + N)})^{H - H_+} \prod_{h=1}^{H_+} \frac{\Gamma(\frac{\alpha\beta}{H} + \beta)\Gamma(N_h + \frac{\alpha\beta}{H})\Gamma(N - N_h + \beta)}{\Gamma(\frac{\alpha\beta}{H})\Gamma(\beta)\Gamma(\frac{\alpha\beta}{H} + \beta + N)}\\
= \frac{H!}{H_0! \prod_{i=1}^{2^N - 1}H_i!} (\frac{\Gamma(\frac{\alpha\beta}{H} + \beta)\Gamma(\frac{\alpha\beta}{H})\Gamma(N + \beta)}{\Gamma(\frac{\alpha\beta}{H})\Gamma(\beta)\Gamma(\frac{\alpha\beta}{H} + \beta + N)})^{H} \prod_{h=1}^{H_+} \frac{\Gamma(N_h + \frac{\alpha\beta}{H})\Gamma(N - N_h + \beta)}{\Gamma(\frac{\alpha\beta}{H}) \Gamma(N + \beta) }\\
= \frac{H!}{H_0! \prod_{i=1}^{2^N - 1}H_i!} (\frac{\Gamma(\frac{\alpha\beta}{H} + \beta)\Gamma(\frac{\alpha\beta}{H})\Gamma(N + \beta)}{\Gamma(\frac{\alpha\beta}{H})\Gamma(\beta)\Gamma(\frac{\alpha\beta}{H} + \beta + N)})^{H} \prod_{h=1}^{H_+} \frac{\Gamma(N_h + \frac{\alpha\beta}{H})\Gamma(N - N_h + \beta)}{\Gamma(\frac{\alpha\beta}{H}) \Gamma(N + \beta) }\\
= \frac{H!}{H_0! \prod_{i=1}^{2^N - 1}H_i!} (\frac{\Gamma(\frac{\alpha\beta}{H} + \beta)\Gamma(N + \beta)}{\Gamma(\beta)\Gamma(\frac{\alpha\beta}{H} + \beta + N)})^{H}  \prod_{h=1}^{H_+} \frac{\frac{\alpha\beta}{H}\prod_{j = 1}^{N_h - 1}(j + \frac{\alpha\beta}{H})\Gamma(\frac{\alpha\beta}{H})\Gamma(N - N_h + \beta)}{\Gamma(\frac{\alpha\beta}{H}) \Gamma(N + \beta) }\\
= \frac{H!}{H_0! \prod_{i=1}^{2^N - 1}H_i!} (\frac{\Gamma(\frac{\alpha\beta}{H} + \beta)\Gamma(\beta) \prod_{j = 0}^{N - 1}(j + \beta)}{\Gamma(\beta)\Gamma(\frac{\alpha\beta}{H} + \beta)\prod_{j = 0}^{N - 1} (j + \frac{\alpha\beta}{H} + \beta) })^{H}  (\frac{\alpha\beta}{H})^{H_+} \prod_{h=1}^{H_+}  \frac{\prod_{j = 1}^{N_h - 1}(j + \frac{\alpha\beta}{H})\Gamma(N - N_h + \beta)}{\Gamma(N + \beta) }\\
= \frac{(\alpha\beta)^{H_+}}{\prod_{i=1}^{2^N - 1}H_i!} \frac{H!}{H_0! H^{H_+}} (\frac{\prod_{j = 1}^{N}(j + \beta - 1)}{\prod_{j = 1}^{N} (j + \beta - 1 + \frac{\alpha\beta}{H}) })^{H}  \prod_{h=1}^{H_+}  \frac{\prod_{j = 1}^{N_h - 1}(j + \frac{\alpha\beta}{H})\Gamma(N - N_h + \beta)}{\Gamma(N + \beta) }\\
= \frac{(\alpha\beta)^{H_+}}{\prod_{i=1}^{2^N - 1}H_i!} \frac{H!}{H_0! H^{H_+}} \prod_{j = 1}^{N} ((\frac{1}{1 + \frac{\alpha\beta}{H(j + \beta - 1)}) })^{H})  \prod_{h=1}^{H_+}  \frac{\prod_{j = 1}^{N_h - 1}(j + \frac{\alpha\beta}{H})\Gamma(N - N_h + \beta)}{\Gamma(N + \beta) }\\
\end{split}
\end{equation}

先の検討を再度考えると以下のようになる。
\begin{equation}
\lim_{H \rightarrow \infty} \frac{H!}{H_0!H^{H_+}} = \lim_{H \rightarrow \infty}(1 + \frac{{H_+}^{H_+ + 1}}{H}) = 1
\end{equation}
この資料の(12)を考慮して、さらに$N_h$が整数なのでガンマ関数の性質から、
\begin{equation}
\begin{split}
\lim_{H \rightarrow \infty} \prod_{j = 1}^{N_h - 1}(j + \frac{\alpha\beta}{H}) = (N_h - 1)! = \Gamma(N_h)
\end{split}
\end{equation}
\begin{equation}
\lim_{H \rightarrow \infty} (\frac{1}{1 + \frac{\alpha\beta}{H(j + \beta - 1)}) })^{H} = exp(-\frac{\alpha\beta}{j + \beta - 1})
\end{equation}
よって、
\begin{equation}
\lim_{H \rightarrow \infty} \prod_{j = 1}^{N}(\frac{1}{1 + \frac{\alpha\beta}{H(j + \beta - 1)}) })^{H} = exp(-(\alpha \sum_{j = 1}^{N}\frac{\beta}{j + \beta - 1}))
\end{equation}
これらを踏まえて(17)を考慮すると、
\begin{equation}
\begin{split}
p([M]) = \frac{(\alpha\beta)^{H_+}}{\prod_{i=1}^{2^N - 1}H_i!} \frac{H!}{H_0! H^{H_+}} \prod_{j = 1}^{N} ((\frac{1}{1 + \frac{\alpha\beta}{H(j + \beta - 1)}) })^{H})  \prod_{h=1}^{H_+}  \frac{\prod_{j = 1}^{N_h - 1}(j + \frac{\alpha\beta}{H})\Gamma(N - N_h + \beta)}{\Gamma(N + \beta) }\\
= \frac{(\alpha\beta)^{H_+}}{\prod_{i=1}^{2^N - 1}H_i!} exp(-(\alpha \sum_{j = 1}^{N}\frac{\beta}{j + \beta - 1})) \prod_{h=1}^{H_+}  \frac{\Gamma(N_h)\Gamma(N - N_h + \beta)}{\Gamma(N + \beta) }\\
\end{split}
\end{equation}
となり、(6.51)が成り立つことがわかる。
\end{document}
