\documentclass{jsarticle}
\usepackage{amsmath,amssymb,amsfonts}
\begin{document}
(6.29),(6.31)について考える。

そもそも、ベクトルのベクトルでの微分を考える。
\begin{equation}
\frac{\partial{\bf f}}{\partial {\bf x}} =
\begin{pmatrix}
\frac{\partial f_1}{\partial x_1} & \cdots & \frac{\partial f_1}{\partial x_n} \\
\vdots & & \vdots\\
\frac{\partial f_m}{\partial x_1} & \cdots & \frac{\partial f_m}{\partial x_n} 
\end{pmatrix},
{\bf f} =
\begin{pmatrix}
f_1 \\
\vdots\\
f_m
\end{pmatrix},
{\bf x} =
\begin{pmatrix}
x_1 \\
\vdots\\
x_n
\end{pmatrix}
\end{equation}
また、
\begin{equation}
\frac{\partial f}{\partial {\bf x}} =
\begin{pmatrix}
\frac{\partial f}{\partial x_1} & \cdots & \frac{\partial f}{\partial x_n} \\
\end{pmatrix}
\end{equation}

平面流では${\bf f}$は(6.28)にあるように以下のようになる。
\begin{equation}
{\bf f}({\bf z}) = {\bf z} + {\bf u} h({\bf w}^T {\bf z} + b)
\end{equation}

これを踏まえて、
\begin{equation}
\begin{split}
\frac{\partial{\bf f}({\bf z})}{\partial {\bf z}} = \frac{\partial {\bf z}}{\partial {\bf z}} + \frac{\partial {\bf u} h({\bf w}^T {\bf z} + b)}{\partial {\bf z}}
=
\begin{pmatrix}
\frac{\partial z_1}{\partial z_1} & \cdots & \frac{\partial z_D}{\partial z_1} \\
\vdots & & \vdots \\
\frac{\partial z_1}{\partial z_D} & \cdots & \frac{\partial z_D}{\partial z_D} 
\end{pmatrix}
+ \frac{{\bf u} \partial h({\bf w}^T {\bf z} + b)}{\partial {\bf z}}
= I + \frac{{\bf u} \partial h({\bf w}^T {\bf z} + b)}{\partial {\bf z}}\\
= I + {\bf u} \frac{\partial h({\bf w}^T {\bf z} + b)}{\partial ({\bf w}^T {\bf z} + b)}\frac{\partial ({\bf w}^T {\bf z} + b)}{\partial {\bf z}}
= I + {\bf u} \frac{\partial h({\bf w}^T {\bf z} + b)}{\partial ({\bf w}^T {\bf z} + b)}{\bf w}^T I
= I + {\bf u} \frac{\partial h({\bf w}^T {\bf z} + b)}{\partial ({\bf w}^T {\bf z} + b)}{\bf w}^T
= I + {\bf u} \psi({\bf z})^T
\end{split}
\end{equation}
これで、元論文の(12)の最初の等号がわかる。

ここで、PRML上巻付録C(C.15)を考えると、
\begin{equation}
|det (\frac{\partial{\bf f}({\bf z})}{\partial {\bf z}})| = |det(I + {\bf u} \psi({\bf z})^T)|
= |1 + {\bf u}^T \psi({\bf z})|
\end{equation}
となり、(6.29)が求まる。

参考だが、行列A,B,C,D(ただし、A,Dは正方行列)に対して、一般に
\begin{equation}
\begin{pmatrix}
A & B\\
C & D
\end{pmatrix}
=
\begin{pmatrix}
A & 0\\
C & I
\end{pmatrix}
\begin{pmatrix}
I & A^{-1}B\\
0 & D-CA^{-1}B
\end{pmatrix}
=
\begin{pmatrix}
I & B\\
0 & D
\end{pmatrix}
\begin{pmatrix}
A - BD^{-1}C & 0\\
D^{-1}C & I
\end{pmatrix}
\end{equation}
となるため、
\begin{equation}
det \begin{pmatrix}
A & B\\
C & D
\end{pmatrix}
=
det \begin{pmatrix}
A & 0\\
C & I
\end{pmatrix}
det
\begin{pmatrix}
I & A^{-1}B\\
0 & D-CA^{-1}B
\end{pmatrix}
=det(AI)det(I(D-CA^{-1}B))
=det(A)det(D-CA^{-1}B)
\end{equation}
となり、同様に
\begin{equation}det
\begin{pmatrix}
A & B\\
C & D
\end{pmatrix}
=
det \begin{pmatrix}
I & B\\
0 & D
\end{pmatrix}
det
\begin{pmatrix}
A - BD^{-1}C & 0\\
D^{-1}C & I
\end{pmatrix}
=det(D)det(A - BD^{-1}C)
\end{equation}
となるため、
\begin{equation}
det(A)det(D-CA^{-1}B) = det(D)det(|A - BD^{-1}C)
\end{equation}

それを踏まえ、(文字が被っているが、)以下をもとに、PRMLの(C.14)を考える。
\begin{equation}
\begin{pmatrix}
I_N & -B\\
A & I_M
\end{pmatrix}
\end{equation}
これから、
\begin{equation}
det(I_N)det(I_M-A^T I_N^{-1}(-B)) = det(I_M)det(I_N + B I_M^{-1} A^T)
\end{equation}
これより、$det(A) = det(A^T)$も考慮すると(行、列の順番で入れ替えていくと、偶置換になり等しくなる。)、
\begin{equation}
det(I_M + A^T B) = det(I_N + B A^T) = det((I_N + B A^T)^T) = det(I_N + A B^T)
\end{equation}
となり、(C.14)が成り立つことがわかる。そのため、(C.15)も成り立つ。

同様に放射状流を考える。
\begin{equation}
{\bf f}({\bf z}) = {\bf z} + \beta h(\alpha, r)({\bf z} -\bar{{\bf z}})
\end{equation}
これより、
\begin{equation}
\begin{split}
\frac{\partial {\bf f}({\bf z})}{\partial {\bf z}} = I + \frac{\partial \beta h(\alpha, r)({\bf z} -\bar{{\bf z}})}{\partial {\bf z}}
= I + \beta \frac{\partial h(\alpha, r)}{\partial r}({\bf z} -\bar{{\bf z}})\frac{\partial r}{\partial {\bf z}}
+ \beta h(\alpha, r)I
\end{split}
\end{equation}
ここで、以下を考える。
\begin{equation}
\frac{\partial r}{\partial {\bf z}} = \frac{\partial \sqrt{\sum ({z_i} - \bar{z_i})^2}}{\partial {\bf z}}=
\begin{pmatrix}
\frac{z_1 - \bar{z_1}}{\sqrt{\sum ({z_i} - \bar{z_i})^2}} & \cdots & \frac{z_n - \bar{z_n}}{\sqrt{\sum ({z_i} - \bar{z_i})^2}}
\end{pmatrix}
= \frac{1}{r}({\bf z}-\bar{\bf z})^T
\end{equation}
よって、
\begin{equation}
\begin{split}
|det(\frac{\partial {\bf f}({\bf z})}{\partial {\bf z}})|
= |det(I + \beta \frac{\partial h(\alpha, r)}{\partial r}({\bf z} -\bar{{\bf z}})\frac{\partial r}{\partial {\bf z}}
+ \beta h(\alpha, r)I)|
= |det((1 + \beta h(\alpha, r))I + \frac{\beta}{r} \frac{\partial h(\alpha, r)}{\partial r}({\bf z} -\bar{{\bf z}})({\bf z} -\bar{{\bf z}})^T)|\\
= |det ((1 + \beta h(\alpha, r))(I + \frac{\beta \frac{\partial h(\alpha, r)}{\partial r}}{r(1 + \beta h(\alpha, r))} ({\bf z} -\bar{{\bf z}})({\bf z} -\bar{{\bf z}})^T))|
= |(1 + \beta h(\alpha, r))^D det(I + \frac{\beta \frac{\partial h(\alpha, r)}{\partial r}}{r(1 + \beta h(\alpha, r))} ({\bf z} -\bar{{\bf z}})({\bf z} -\bar{{\bf z}})^T)|\\
= |(1 + \beta h(\alpha, r))^D (1 + \frac{\beta \frac{\partial h(\alpha, r)}{\partial r}}{r(1 + \beta h(\alpha, r))} ({\bf z} -\bar{{\bf z}})^T({\bf z} -\bar{{\bf z}}))|
= |(1 + \beta h(\alpha, r))^D (1 + \frac{\beta h'(\alpha, r)r}{1 + \beta h(\alpha, r)} )|\\
= |(1 + \beta h(\alpha, r))^{D-1} (1 + \beta h(\alpha, r) + \beta h'(\alpha, r)r)|
\end{split}
\end{equation}
よって、(6.31)が求まる。

(6.31)を厳密に見ると、上記は絶対値を取っているが、(6.31)は絶対値を取っていない。
元論文を読むと、(14)では$h = \frac{1}{r + \alpha}, \alpha > 0, r \geq 0$という条件付がされている。

hは微分可能なので、$1 + \beta h(\alpha, r), (1 + \beta h(\alpha, r) + \beta h'(\alpha, r)r)$も微分可能になり、ここが0になるとヤコビアンが0になり、条件を満たさない。

\begin{equation}
\begin{split}
\lim_{r \rightarrow \infty} (1 + \beta h(\alpha, r)) = 1\\
\lim_{r \rightarrow \infty} (1 + \beta h(\alpha, r) + \beta h'(\alpha, r)r) = 1
\end{split}
\end{equation}
なので、絶対値を取らなくても、それぞれの項が正になることがわかる。

よって、
\begin{equation}
\begin{split}
|det(\frac{\partial {\bf f}({\bf z})}{\partial {\bf z}})|
= (1 + \beta h(\alpha, r))^{D-1} (1 + \beta h(\alpha, r) + \beta h'(\alpha, r)r)
\end{split}
\end{equation}
\end{document}
