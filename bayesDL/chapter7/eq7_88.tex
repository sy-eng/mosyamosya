\documentclass{jsarticle}
\usepackage{amsmath,amssymb,amsfonts}

\begin{document}
(7.88)を考える。

(7.71)と(7.73)を見比べる。$z_j^{(0)}(x)$に相当するものを考えると、$x_j$となる。
ただし、$x^T = (x_1, x_2, \cdots, x_{H_0})$となる。

(7.76)を振り返ると$x_j$値やその分布、$H_0$によらず、(7.76)が成り立つ。つまり、$m_i^{(1)}(x) = 0$

(7.80)を振り返ると、(7.76)が成り立っているので、3つめの等号まで(7.80)が成り立っている。
ただし、$C_j(x, x') = \mathbb{E}[\{ z_j^{(0)}(x) z_j^{(0)}(x')] = \mathbb{E}[ x_j x'_j ]$と置く。
導出は(7.80)と同様になっている。
$l = 1$でないときは、任意のjで$C(x) = C_j(x) = \mathbb{E}[ z_j^{(l -1)}(x) z_j^{(l -1)}(x')]$としているが、
$z_j^{(l-1)}(x)$の生成方法を考慮すると、jによらず、同分布になっていると言える。

一方で、$x_j$に関して、$x_j$になることがわかっているので、$p_j(y) = \delta_{x_j, y}$となる。

よって、
\begin{equation}
C_j(x, x') = \mathbb{E}[ x_j x'_j] = \int yz p_j(y)p_j(z) dydz = x_j x'_j
\end{equation}
(7.80)に代入すると、
\begin{equation}
\begin{split}
k^{(1)}(x, x') = k_i^{(1)}(x, x') = \sum_j^{H_0} \mathbb{E}[\{ w_{i, j}^{(1)}\}^2]\mathbb{E}[x_j x'_j] + \mathbb{E}[\{ b_j^{(1)}\}^2]
= \sum_j^{H_0} v_w^{(1)} x_j x'_j + v_b^{(1)}\\
= v_w^{(1)} \sum_j^{H_0} x_j x'_j + v_b^{(1)}
= v_w^{(1)} x^T x' + v_b^{(1)}
\end{split}
\end{equation}
(7.78)のように$v_w^{(1)} = \frac{\hat{v}_w^{(1)}}{H_0}$とすると、(7.88)のように、
\begin{equation}
k^{(1)}(x, x') = v_w^{(1)} x^T x' + v_b^{(1)}
= v_b^{(1)} + v_w^{(1)} \frac{x^T x'}{H_0}
\end{equation}
\end{document}
