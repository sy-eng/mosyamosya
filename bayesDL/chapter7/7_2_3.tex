\documentclass{jsarticle}
\usepackage{amsmath,amssymb,amsfonts}
\usepackage[dvipdfmx]{hyperref}
\usepackage{pxjahyper}
\begin{document}
7.2.3に関して、確認する。

(7.36)の確率分布を仮定する。
累積密度関数は一般には確率分布になっていないが、$y_n$が$\{-1, 1\}$なので、
確率分布になる。

$p(Y | F, X) = p(Y | F) = \prod_{n = 1}^N p(y_n | f_n)$を仮定すると、$p(Y | X)$を定数として、
\begin{equation}
p(F | Y, X)p(Y | X) = p(Y | F, X)p(F | X) = p(F | X)\prod_{n = 1}^N p(y_n | f_n)
\end{equation}
となり、(7.37)が成り立つ。

パラメータはそれぞれの$\tilde{\mu}_n, {\tilde{\sigma}_n}^2$とすると、一部定数項があるとして、
(4.84)を参考に、(7.38)のように仮定する。このとき、(4.82)と(4.84)式との対比を考えると、
$f_n(\theta)=p(y_n|f_n)$、$\tilde{f}_n(\theta) = t(f_n | \tilde{\mu}_n, {\tilde{\sigma}^2}_n)$となる。
一部、fの記号が混在しているので注意のこと。tは(7.39)のようにガウス分布として、
p.199の下から2行を考慮すると$p(F | X) = \mathcal{N}(F | 0, K_\beta)$になり、
(4.85)と同様に平方展開を行うと、(7.40)-(7.42)が求まる。なお、Fと$f_n$の関係を考えると、
$p(Y | F)$は$\tilde{\mu}, \tilde{\Sigma}$を持つ多次元ガウス分布となる。

この分布でうまく置換し、$f_i$が左上になるようにする。すると、(3.31)で$x_i = \{ f_i \}$、
残りの要素が$x_2$とみなせ、(3.31)より、(7.43)の$\mu_i, {\sigma_i}^2$が計算でき、
(7.43)のパラメータの初期値がもとまる。
(4.86)のように$\tilde{f}_i$に相当する、$t$で割る。(4.85)のように平方完成すると、
(7.44)-(7.46)が求まる。

次に、(4.87)のように、fに相当するものをかけて、rを求める。

その後、(4.65),(4.66)から、(7.43)のパラメータを更新する。
(4.65),(4.66)の計算には$Z$の微分が必要になる。(7.47)は
\begin{equation}
r(f_i) = \frac{1}{Z_i}\Phi(y_i | f_i)\mathcal{N}(f_i | \mu_{\backslash i}, {\sigma_{\backslash i}}^2)
\end{equation}
なので、$Z_i$は(4.78)(4.79)やA.3を参考にすると、(4.79)(4.80),(A.44)のように、
\begin{equation}
{Z_i} = \Phi(a_i) (a_i = \frac{y_n \mu_{\backslash i}}{\sqrt{1 + {y_i}^2 \sigma_{\backslash i}^2}}
= \frac{y_n \mu_{\backslash i}}{\sqrt{1 + \sigma_{\backslash i}^2}})
\end{equation}
(4.65)を考慮すると、平均の更新は$\mu_i$が$\mu_{\backslash i}$に、$\mu_{i+1}$が
$\hat{\mu}_i$に対応するので、
\begin{equation}
\begin{split}
\hat{\mu}_i = \mu_{\backslash i} + {\sigma_{\backslash i}}^2 \frac{\partial}{\partial \mu_{\backslash i}} ln \, Z_i
= \mu_{\backslash i} + {\sigma_{\backslash i}}^2 \frac{\partial}{\partial Z_i} ln \, Z_i \frac{\partial Z_i}{\partial a_i}\frac{\partial a_i}{\partial \mu_{\backslash i}}
= \mu_{\backslash i} + {\sigma_{\backslash i}}^2 \frac{1}{Z_i} \frac{\partial}{\partial a_i}\int_{-\infty}^{a_i} \mathcal{N}(x | 0, 1)dx \frac{y_n}{\sqrt{1 + \sigma_{\backslash i}^2}}\\
= \mu_{\backslash i} + {\sigma_{\backslash i}}^2 \frac{1}{\Phi(a_i)} \mathcal{N}(a_i | 0, 1) \frac{y_n}{\sqrt{1 + \sigma_{\backslash i}^2}}
= \mu_{\backslash i} + \frac{y_n {\sigma_{\backslash i}}^2  \mathcal{N}(a_i | 0, 1)}{\Phi(a_i)\sqrt{1 + \sigma_{\backslash i}^2}}
\end{split}
\end{equation}
となり、(7.48)が求まる。
(4.66)を考慮すると、平均と同様に
\begin{equation}
\begin{split}
{\hat{\sigma}_i}^2 = {\sigma_{\backslash i}}^2 - ({\sigma_{\backslash i}}^2)^2\{ (\frac{1}{\Phi(a_i)} \mathcal{N}(a_i | 0, 1) \frac{y_n}{\sqrt{1 + \sigma_{\backslash i}^2}})^2 - 2 \frac{\partial}{\partial Z_i} ln \, Z_i \frac{\partial Z_i}{\partial a_i}\frac{\partial a_i}{\partial {\sigma_{\backslash i}}^2}\}\\
= {\sigma_{\backslash i}}^2 - {\sigma_{\backslash i}}^4\{ \frac{{y_i}^2\mathcal{N}(a_i | 0, 1)^2}{\Phi(a_i)^2(1 + \sigma_{\backslash i}^2)} - 2 \frac{\partial}{\partial Z_i} ln \, Z_i \frac{\partial Z_i}{\partial a_i}\frac{\partial a_i}{\partial {\sigma_{\backslash i}}^2}\}\\
= {\sigma_{\backslash i}}^2 - {\sigma_{\backslash i}}^4\{ \frac{{y_i}^2\mathcal{N}(a_i | 0, 1)^2}{\Phi(a_i)^2(1 + \sigma_{\backslash i}^2)} - 2 \frac{1}{\Phi(a_i)} \mathcal{N}(a_i | 0, 1)(-\frac{y_i \mu_{\backslash i}}{2(\sqrt{1 + \sigma_{\backslash i}^2})^3})\}\\
= {\sigma_{\backslash i}}^2 - \frac{{\sigma_{\backslash i}}^4 \mathcal{N}(a_i | 0, 1)}{\Phi(a_i)(1 + \sigma_{\backslash i}^2)}\{ \frac{{y_i}^2\mathcal{N}(a_i | 0, 1)}{\Phi(a_i)} + a_i\}
= {\sigma_{\backslash i}}^2 - \frac{{\sigma_{\backslash i}}^4 \mathcal{N}(a_i | 0, 1)}{\Phi(a_i)(1 + \sigma_{\backslash i}^2)}\{ a_i + \frac{\mathcal{N}(a_i | 0, 1)}{\Phi(a_i)}\}
\end{split}
\end{equation}
最後に$y_i = \{ -1, 1\}$に注意する。(7.51),(7.52)は、(7.45),(7.46)と同様に考える。
\end{document}