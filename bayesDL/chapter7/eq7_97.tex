\documentclass{jsarticle}
\usepackage{amsmath,amssymb,amsfonts}

\begin{document}
(7.97)-(7.99)を考える。

(7.95)にあるように、
\begin{equation}
\mathcal{J}(m) = (sin\,\theta)^{2m+1} ((\frac{\partial}{\partial(cos\,\theta)})^{m}\frac{\pi - \theta}{sin\,\theta})
\end{equation}

(7.97)に関してはm=0で容易。
\begin{equation}
\mathcal{J}(0) = sin\,\theta (\frac{\pi - \theta}{sin\,\theta}) = \pi - \theta
\end{equation}

まず、$z = cos\,\theta$と置く。すると、
\begin{equation}
\frac{\partial z}{\partial \theta} = -sin\,\theta
\end{equation}
よって、
\begin{equation}
\frac{\partial \theta}{\partial z} = -\frac{1}{sin\,\theta} =  -\frac{1}{\sqrt{1-z^2}}
\end{equation}
なお、$cos\,\theta$の定義から$sin\,\theta > 0$となっていることに注意する。

(7.98)にあるようにm=1の場合を考える。
\begin{equation}
\mathcal{J}(1) = (sin\,\theta)^{3} (\frac{\partial}{\partial(cos\,\theta)}\frac{\pi - \theta}{sin\,\theta})
\end{equation}
上記の微分のところだけを考慮する。
\begin{equation}
\begin{split}
\frac{\partial}{\partial(cos\,\theta)}\frac{\pi - \theta}{sin\,\theta} = \frac{\partial}{\partial z}\frac{\pi - \theta}{\sqrt{1-z^2}}
= -\frac{\partial \theta}{\partial z}\frac{1}{\sqrt{1-z^2}} + (-\frac{1}{2})(-2z)\frac{1}{\sqrt{1-z^2}^3}(\pi - \theta)\\
= \frac{1}{\sqrt{1-z^2}^2} + \frac{z}{\sqrt{1-z^2}^3}(\pi - \theta)
= \frac{1}{1-z^2} + \frac{z}{\sqrt{1-z^2}^3}(\pi - \theta)
\end{split}
\end{equation}
よって、以下のように(7.98)が成立する。
\begin{equation}
\begin{split}
\mathcal{J}(1) = (sin\,\theta)^{3} (\frac{1}{\sqrt{1-z^2}^2} + \frac{z}{\sqrt{1-z^2}^3}(\pi - \theta))
=(sin\,\theta)^{3} (\frac{1}{(sin\,\theta)^2} + \frac{cos\,\theta}{(sin\,\theta)^3}(\pi - \theta))\\
=sin\,\theta + (\pi - \theta))cos\,\theta
\end{split}
\end{equation}

同様に(7.99)にある、m=2の場合を考える。
\begin{equation}
\mathcal{J}(2) = (sin\,\theta)^{5} ((\frac{\partial}{\partial(cos\,\theta)})^2 \frac{\pi - \theta}{sin\,\theta})
\end{equation}
上記の微分のところだけを考慮する。
\begin{equation}
\begin{split}
(\frac{\partial}{\partial(cos\,\theta)})^2 \frac{\pi - \theta}{sin\,\theta} = (\frac{\partial}{\partial z})^2 \frac{\pi - \theta}{\sqrt{1-z^2}}
= \frac{\partial}{\partial z}(\frac{1}{1-z^2} + \frac{z}{\sqrt{1-z^2}^3}(\pi - \theta))\\
= (-1)(-2z)\frac{1}{(1-z^2)^2} + (\frac{1}{\sqrt{1-z^2}^3} + (-\frac{3}{2})(-2z)\frac{z}{\sqrt{1-z^2}^5})(\pi - \theta) + \frac{z}{\sqrt{1-z^2}^3}(- \frac{\partial \theta}{\partial z})
= \frac{2z}{(1-z^2)^2} + (\frac{(1-z^2) + 3z^2}{\sqrt{1-z^2}^5})(\pi - \theta) + \frac{z}{\sqrt{1-z^2}^3}(- \frac{\partial \theta}{\partial z})
= \frac{2z}{(1-z^2)^2} + (\frac{1 + 2z^2}{\sqrt{1-z^2}^5})(\pi - \theta) + \frac{z}{(1-z^2)^2}\\
= \frac{3z}{(1-z^2)^2} + (\frac{1 + 2z^2}{\sqrt{1-z^2}^5})(\pi - \theta)
\end{split}
\end{equation}

よって、以下のように(7.99)が成立する。
\begin{equation}
\begin{split}
\mathcal{J}(2) = (sin\,\theta)^{5} (\frac{3z}{(1-z^2)^2} + (\frac{1 + 2z^2}{\sqrt{1-z^2}^5})(\pi - \theta))
=(sin\,\theta)^{5} (\frac{3cos\,\theta}{(sin\,\theta)^4} + (\frac{1 + 2cos^2\,\theta}{(sin\,\theta)^5})(\pi - \theta))\\
=3cos\,\theta \, sin\,\theta + (\pi - \theta)(1 + 2cos^2\,\theta)
\end{split}
\end{equation}

\end{document}
