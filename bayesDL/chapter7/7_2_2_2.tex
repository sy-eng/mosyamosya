\documentclass{jsarticle}
\usepackage{amsmath,amssymb,amsfonts}
\usepackage[dvipdfmx]{hyperref}
\usepackage{pxjahyper}
\begin{document}
7.2.2.2のラプラス近似を用いた予測分布の近似について詳細を確認する。
この付近についてはPRMLの6.4、特に6.4.6で同様の議論されている。

一般に、
\begin{equation}
p(f_* | Y, X, x_*) = \int p(f_* | Y, X, x_*, F) p(F | Y, X, x_*)dF
\end{equation}
となるが、ガウス過程の仮定より、
\begin{equation}
p(f_* | Y, X, x_*, F) = p(f_* | X, x_*, F)
\end{equation}
FはX,Yのみから推定されるので、
\begin{equation}
p(F | Y, X, x_*) = p(F | Y, X)
\end{equation}
これをまとめると、
\begin{equation}
p(f_* | Y, X, x_*) = \int p(f_* | X, x_*, F) p(F | Y, X)dF
\end{equation}
となる。

ここで、(7.23)を用いて、
\begin{equation}
p(F | Y, X) \approx q(F | X, Y) = \mathcal{N}(F | \hat{F}, \Lambda^{-1})
\end{equation}
よって、$p(f_* | Y, X, x_*)$も近似し、(7.31)のように、
\begin{equation}
p(f_* | Y, X, x_*) \approx q(f_* | Y, X, x_*) \equiv \int p(f_* | X, x_*, F) q(F | X, Y)dF
\end{equation}
となる。

ここで、(7.20)の下も考慮し、ガウス過程から$p(x_*, X | f_*, F)$を考える。
\begin{equation}
p(f_*, F | x_*, X) = \mathcal{N} \left( \left[
\begin{matrix}
f_*\\
F
\end{matrix}
\right] | 0, 
\left[
\begin{matrix}
k_\beta(x_*, x_*) & {k_*}^T\\
k_* & {K_\beta}^{-1}
\end{matrix}
\right] \right)
\end{equation}
これは、(7.7)と同形状で、(7.8)のように
\begin{equation}
p(f_* | x_*, X, F) = \mathcal{N} ( f_* | {k_*}^T {K_{\beta}}^{-1}F, k_\beta(x_*, x_*)
- {k_*}^T {K_{\beta}}^{-1} k_*)
\end{equation}
よって、(A.24)も考慮して、
\begin{equation}
q(f_* | Y, X, x_*) = \int \mathcal{N} ( f_* | {k_*}^T {K_{\beta}}^{-1}F, k_\beta(x_*, x_*)
- {k_*}^T {K_{\beta}}^{-1} k_*) \mathcal{N}(F | \hat{F}, \Lambda^{-1}) dF
= \mathcal{N} ( f_* | \mu_*, {\sigma_*}^2)
\end{equation}
ここで、
\begin{equation}
\mu_* = {k_*}^T {K_{\beta}}^{-1} \hat{F}\\
\end{equation}
$K_\beta$が対称行列であることも踏まえ、
\begin{equation}
\begin{split}
{\sigma_*}^2 = k_\beta(x_*, x_*) - {k_*}^T {K_{\beta}}^{-1} k_* + {k_*}^T {K_{\beta}}^{-1} \Lambda^{-1}({k_*}^T {K_{\beta}}^{-1})^T\\
= k_\beta(x_*, x_*) - {k_*}^T {K_{\beta}}^{-1} k_* + {k_*}^T {K_{\beta}}^{-1} \Lambda^{-1} {K_{\beta}}^{-1} k_*\\
= k_\beta(x_*, x_*) - {k_*}^T ({K_{\beta}}^{-1} - {K_{\beta}}^{-1} \Lambda^{-1} {K_{\beta}}^{-1} )k_*
\end{split}
\end{equation}
さて、(7.25)を見ると(7.28)は$\Lambda$を表していることがわかる。
(7.30)を復習すると、これは対角行列になっており、これを-Wとすると、
\begin{equation}
\Lambda = W + {K_\beta}^{-1}
\end{equation}
これを踏まえると、
\begin{equation}
{\sigma_*}^2 = k_\beta(x_*, x_*) - {k_*}^T ({K_{\beta}}^{-1} - {K_{\beta}}^{-1} (W + {K_\beta}^{-1})^{-1} {K_{\beta}}^{-1} )k_*
\end{equation}
この式のカッコの中を(A.1)と見比べると、$A = K_{\beta}, U = I, B = W^{-1}, V = I$の対応がつくことがわかり、その結果、
\begin{equation}
{\sigma_*}^2 = k_\beta(x_*, x_*) - {k_*}^T ({K_\beta} + W^{-1})^{-1} {K_{\beta}}^{-1} )k_*
\end{equation}
となることがわかる。(本に誤りがあるのでは??PRMLでは少し、変数の割り当て方が違うが、これと同様になっている。)
\end{document}
