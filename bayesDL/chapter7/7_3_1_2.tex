\documentclass{jsarticle}
\usepackage{amsmath,amssymb,amsfonts}
\usepackage[dvipdfmx]{hyperref}
\usepackage{pxjahyper}
\begin{document}
7.3.1.2を確認する。

論文[115]の2章を見てみる。

(4)式にあるように一般的な点を$\{ x, y, (f) \}$を考えたときにガウス過程では任意の$f$たちの
組み合わせが平均0,分散Kのガウス分布になるので$\{ F_Z, f, Z, x \}$を考えると、
(A.7),(A.3)を考慮して、(7.9)と同様に
\begin{equation}
p(f | x, Z, F_Z) = \mathcal{N}(f | k_{xZ} {K_{ZZ}}^{-1} F_Z, k_{xx} - k_{xZ} {K_{ZZ}}^{-1} k_{Zx})
\end{equation}
となる。

一般に
\begin{equation}
p(y | f) = \mathcal{N}(y | f, \sigma^2)
\end{equation}
なので、本の(A.24)を踏まえると、
\begin{equation}
p(y | x, Z, F_Z) = \mathcal{N}(y | k_{xZ} {K_{ZZ}}^{-1} F_Z, k_{xx} - k_{xZ} {K_{ZZ}}^{-1} k_{Zx} + \sigma^2)
\end{equation}
となり、[115]の(4)式になる。

さて、(7.63)も考慮すると、
\begin{equation}
\begin{split}
p(y_* | Y) = p(y_* | x_*, X, Y) = p(y_* | x*, X, Y, Z) = \int p(y_*, F_Z | x*, X, Y, Z) dF_Z\\
= \int p(y_* | F_Z, x*, X, Y, Z) p(F_Z | x*, X, Y, Z) dF_Z 
\approx \int p(y_* | F_Z, x*, Z) p(F_Z | x*, X, Y, Z) dF_Z\\
= \int p(y_* | F_Z, x*, Z) p(F_Z | x*, X, Y) dF_Z
\approx \int p(y_* | F_Z, x*, Z) q(F_Z) dF_Z\\
= \int \mathcal{N}(y | k_{xZ} {K_{ZZ}}^{-1} F_Z, k_{xx} - k_{xZ} {K_{ZZ}}^{-1} k_{Zx} + \sigma^2 \mathcal{N}(F_Z | \mu, \Sigma)dF_Z\\
= \mathcal{N}(y | k_{xZ} {K_{ZZ}}^{-1} \mu, k_{xx} - k_{xZ} {K_{ZZ}}^{-1} k_{Zx} + \sigma^2 + k_{xZ} {K_{ZZ}}^{-1} \Sigma {K_{ZZ}}^{-1} k_{Zx})
\end{split}
\end{equation}
となる。

なお、定義を踏まえると、${k_{xZ}}^T = k_{Zx}, {{K_{ZZ}}^{-1}}^T = {K_{ZZ}}^{-1}$。

形は異なるが、(7.67)が求まった?

($p(f_* | Y)$と考えると一致する。)

論文[115]の2章について、少し詳しく確認する。(4)式をスカラのx,yでなく、ベクトルとして
考え、各事象は独立だと考えると、以下のように(5)式になる。
\begin{equation}
p(Y | X, Z, F_Z) = \prod_{n = 1}^N p(y_n | x_n, Z, F_Z) 
= \mathcal{N}(Y | K_{XZ}{K_{ZZ}}^{-1}K_{ZX} F_Z, \Lambda + \sigma^2 I)
\end{equation}
ここで、
\begin{equation}
\begin{split}
\Lambda = diag(\lambda_n)\\
\lambda_n = k_{x_n, x_n} - k_{x_n Z}{K_{ZZ}}^{-1}k_{Z x_n}
\end{split}
\end{equation}

そもそも、$\{ Z, F_Z \}$に関しても、ガウス過程になっており、$F_Z$はZのみに依存するので、
\begin{equation}
p(F_Z | Z) = p(F_Z | X, Z) = \mathcal{N}(F_Z | 0, K_{ZZ})
\end{equation}

一般に以下が成り立つ。
\begin{equation}
p(F_Z | X, Y, Z) = \frac{p(F_Z, Y | X, Z)}{p(Y | X, Z)} = \frac{p(Y | X, Z, F_Z)p(F_Z | X, Z)}{p(Y | X, Z)}
\end{equation}
$F_Z$を変数、それ以外を定数Cとして、以下のようになる。(正規分布の共分散行列は対称行列であること、スカラになっている部分があることに注意する。)
\begin{equation}
\begin{split}
ln \, p(F_Z | X, Y, Z) = -\frac{1}{2} (F_Z^T {K_{ZZ}}^{-1} F_Z + (Y - K_{XZ} {K_{ZZ}}^{-1} F_Z)^T (\Lambda + \sigma^2 I)^{-1}(Y - K_{XZ} {K_{ZZ}}^{-1} F_Z)) + C\\
= -\frac{1}{2} (F_Z^T ({K_{ZZ}}^{-1} + {K_{ZZ}}^{-1} K_{ZX} (\Lambda + \sigma^2 I)^{-1} K_{XZ} {K_{ZZ}}^{-1})F_Z - 2F_Z {K_{ZZ}}^{-1} K_{ZX} (\Lambda + \sigma^2 I)^{-1}Y) + C\\
= -\frac{1}{2} (F_Z^T {K_{ZZ}}^{-1}(K_{ZZ} + K_{ZX} (\Lambda + \sigma^2 I)^{-1} K_{XZ}){K_{ZZ}}^{-1}F_Z - 2F_Z {K_{ZZ}}^{-1} K_{ZX} (\Lambda + \sigma^2 I)^{-1}Y) + C\\
= -\frac{1}{2} (F_Z - {K_{ZZ}} Q^{-1} K_{ZX}(\Lambda + \sigma^2 I)^{-1}Y)^T {K_{ZZ}}^{-1} Q {K_{ZZ}}^{-1} (F_Z - {K_{ZZ}} Q^{-1} K_{ZX}(\Lambda + \sigma^2 I)^{-1}Y) + C
\end{split}
\end{equation}
これは確率分布になっているため、Cで適当に正規化され、(7)式のようにガウス分布になる。
\begin{equation}
p(F_Z | X, Y, Z)
= \mathcal{N}(F_Z | {K_{ZZ}} Q^{-1} K_{ZX}(\Lambda + \sigma^2 I)^{-1}Y, {K_{ZZ}} Q^{-1} {K_{ZZ}})
\end{equation}

(8)式に関して、近似が入っているような気がするが、
\begin{equation}
\begin{split}
p(y_* | x_*, Y, X, Z) = \int p(y_* | F_Z, x_*, Y, X, Z)p(F_Z | x_*, Y, X, Z) dF_Z\\
\approx \int p(y_* | F_Z, x_*, Z)p(F_Z | x_*, Y, X, Z) dF_Z 
= \int p(y_* | F_Z, x_*, Z)p(F_Z | Y, X, Z) dF_Z\\
= \int \mathcal{N}(y_* | k_{x_* Z} {K_{ZZ}}^{-1} F_Z, k_{x_* x_*} - k_{x_* Z} {K_{ZZ}}^{-1} k_{Zx_*} + \sigma^2)\mathcal{N}(F_Z | {K_{ZZ}} Q^{-1} K_{ZX}(\Lambda + \sigma^2 I)^{-1}Y, {K_{ZZ}} Q^{-1} {K_{ZZ}}) dF_Z\\
=\mathcal{N}(y_* | k_{x_* Z} Q^{-1} K_{ZX}(\Lambda + \sigma^2 I)^{-1}Y, k_{x_* x_*} - k_{x_* Z} {K_{ZZ}}^{-1} k_{Zx_*} + \sigma^2 + k_{x_* Z} {K_{ZZ}}^{-1} {K_{ZZ}} Q^{-1} {K_{ZZ}} {K_{ZZ}}^{-1} k_{Z x_*})\\
=\mathcal{N}(y_* | k_{x_* Z} Q^{-1} K_{ZX}(\Lambda + \sigma^2 I)^{-1}Y, k_{x_* x_*} - k_{x_* Z} {K_{ZZ}}^{-1} k_{Zx_*} + \sigma^2 + k_{x_* Z} Q^{-1} k_{Z x_*})\\
=\mathcal{N}(y_* | k_{x_* Z} Q^{-1} K_{ZX}(\Lambda + \sigma^2 I)^{-1}Y, k_{x_* x_*} - k_{x_* Z} ({K_{ZZ}}^{-1} - Q^{-1})k_{Zx_*} + \sigma^2)
\end{split}
\end{equation}
となり、(8)式が求まる。
\end{document}
