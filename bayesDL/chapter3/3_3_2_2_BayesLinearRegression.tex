\documentclass{jsarticle}
\usepackage{amsmath,amssymb,amsfonts}

\begin{document}
3.3.2.2の計算を確認する。

予測分布は以下の式で求まる。
\begin{equation}
p(y_* | x_*, Y, X) = \int p(y_*, w | x_*, Y, X) dw = \int p(y_* | x_*, w)p(w | Y, X) dw
\end{equation}
なお、観測は独立であるので、$p(y_* | x_*, w) = p(y_* | x_*, w, Y, X), p(w | Y, X) = p(w | x_*, Y, X)$である。

本では、ベイズの定理を用いて、(3.74)を求めることになっているが、面倒なので、(A.24)を利用することで(3.76),(3.77)を求める。(3.74)を直接求めるには(A.24)の導出を参考にすれば良い。

(A.14),(A.15)が(3.71),(3.67)に相当する。

(A.14)と(3.71)で$x = w, \mu = \hat{\mu}, \Sigma_x = \hat{\Sigma}$, (A.15),(3.67)で$y = y_*, W = {\phi(x_*)}^T, b = 0, \Sigma_y = {\sigma_y}^2$と対応する。

そうすると、(A.24)を見ると、
\begin{equation}
\mu_*(x_*) = {\phi(x_*)}^T \hat{\mu} = {\hat{\mu}}^T \phi(x_*)
\end{equation}
($y_*$がスカラなので、$\mu_*(x_*)$もスカラ。そのため、転置しても等しい。)
\begin{equation}
{\sigma_*}^2(x_*) = {\sigma_y}^2 + {\phi(x_*)}^T \hat{\Sigma} \phi(x_*)
\end{equation}
よって、(3.76),(3.77)が求まった。

(3.75)を具体的に書き下すことで、(3.74)が正しいことを確認する。
\begin{equation}
\begin{split}
ln p(y_* | x_*, X, Y) = -\frac{1}{2}{\sigma_*}^{-2}(x_*)(y_* - \mu_*(x_*))^2 + c
= -\frac{1}{2}{\sigma_*}^{-2}(x_*)({y_*}^2 - 2\mu_*(x_*) y_*) + c\\
= -\frac{1}{2}(({\sigma_y}^2 + {\phi(x_*)}^T \hat{\Sigma} \phi(x_*))^{-1}{y_*}^2 - 2({\sigma_y}^2 + {\phi(x_*)}^T \hat{\Sigma} \phi(x_*))^{-1} {\phi(x_*)}^T \hat{\mu} y_*) + c
\end{split}
\end{equation}
最右辺の第1項に対して、(A.1)を、第2項に対して、(A.2)を変形したものを適用する。

(A.1)で$A=\sigma_y^2, U = {\phi(x_*)}^T, B = \hat{\Sigma}, V = \phi(x_*)$とする。(A.2)で$P={\sigma_y}^{-2}, B = \phi(x_*), R = {\hat{\Sigma}}^{-1}$として、両辺の右側からRをかけて、左辺を利用する。

すると、
\begin{equation}
\begin{split}
ln p(y_* | x_*, X, Y)
= -\frac{1}{2}(({\sigma_y}^{-2} - {\sigma_y}^{-2}{\phi(x_*)}^T({\hat{\Sigma}}^{-1} + \phi(x_*) {\sigma_y}^{-2} {\phi(x_*)}^T)^{-1}\phi(x_*){\sigma_y}^{-2}){y_*}^2\\
- 2({\sigma_y}^{-2} {\phi(x_*)}^T ({\phi(x_*)}^T {\sigma_y}^{-2} \phi(x_*) + {\hat{\Sigma}}^{-1})^{-1} {\hat{\Sigma}}^{-1} \hat{\mu} y_*) + c\\
= -\frac{1}{2}(({\sigma_y}^{-2} - {\sigma_y}^{-4}{\phi(x_*)}^T({\sigma_y}^{-2} \phi(x_*) {\phi(x_*)}^T + {\hat{\Sigma}}^{-1})^{-1}\phi(x_*)){y_*}^2\\
- 2({\phi(x_*)}^T {\sigma_y}^{-2} ({\sigma_y}^{-2} {\phi(x_*)}^T \phi(x_*) + {\hat{\Sigma}}^{-1})^{-1} {\hat{\Sigma}}^{-1} \hat{\mu} y_*) + c
\end{split}
\end{equation}
(スカラは順番を入れ替えても良い。)

よって、(3.74)が成立していることがわかる。

\end{document}
