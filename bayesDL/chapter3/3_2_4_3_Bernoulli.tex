\documentclass{jsarticle}
\usepackage{amsmath,amssymb,amsfonts}

\begin{document}
3.2.4.3を膨らませてみる。

そもそも、本にあるように、(3.39)のパラメータの定義をそのまま利用する。
すると、
\begin{equation}
\mu = \frac{e^\eta}{1+e^\eta}
\end{equation}
ベータ分布(3.49)を(3.11)のように変数変換すると(3.50)にあるように、以下のようになる。
\begin{equation}
\begin{split}
Beta_\eta (\eta | \lambda_1, \lambda_2) = Beta(\mu | \alpha, \beta) \frac{d\mu}{d\eta} = exp((\alpha - 1) ln \mu + (\beta - 1) ln (1 - \mu) + ln \frac{\Gamma(\alpha + \beta}{\Gamma(\alpha)\Gamma(\beta)})(-\frac{e^\eta}{(1+e^\eta)^2} + \frac{e^\eta}{1+e^\eta})\\
= exp((\alpha - 1) ln \frac{e^\eta}{1+e^\eta} + (\beta - 1) ln \frac{1}{1+e^\eta} + ln \frac{\Gamma(\alpha + \beta}{\Gamma(\alpha)\Gamma(\beta)})(\frac{e^\eta}{(1+e^\eta)^2}) \\
= exp(\eta (\alpha - 1) - (\alpha + \beta - 2)ln (1+e^\eta) + ln \frac{\Gamma(\alpha + \beta}{\Gamma(\alpha)\Gamma(\beta)})(\frac{e^\eta}{(1+e^\eta)^2})
= exp(\eta \alpha - (\alpha + \beta)ln (1+e^\eta) + ln \frac{\Gamma(\alpha + \beta)}{\Gamma(\alpha)\Gamma(\beta)})
\end{split}
\end{equation}
(3.39)より、$a(\eta) = ln(1+e^\eta)$なので、上記の式は(3.51)と一致する。

上記の式を(3.45)の式と見比べると、(3.51)の式が出てくる。

(3.52)に関して、(3.47)を踏まえると
\begin{equation}
\begin{split}
p(\mu | {\bf X}) = Beta_\eta(\eta | \hat{\lambda_1}, \hat{\lambda_2})
= C exp(\eta (\alpha + \sum_{n = 1}^N x_n) - a(\eta)(\alpha + \beta + N))\\
= C exp((\alpha + \sum_{n = 1}^N x_n - 1)ln (e^\eta) - ln (1+e^\eta)(\alpha + \beta + N -  2))(\frac{e^\eta}{(1+e^\eta)^2})\\
= C exp((\alpha + \sum_{n = 1}^N x_n - 1)ln \frac{e^\eta}{1+e^\eta} + ln \frac{1}{1+e^\eta}(\beta + N - \sum_{n = 1}^N x_n - 1))(\frac{e^\eta}{(1+e^\eta)^2})\\
= C exp((\alpha + \sum_{n = 1}^N x_n - 1)ln \mu + ln (1 - \mu)(\beta + N - \sum_{n = 1}^N x_n - 1))\frac{d\eta}{d\mu} = Beta(\mu | \hat{\alpha}, \hat{\beta})\frac{d\eta}{d\mu}\\
\end{split}
\end{equation}
(自然パラメータのときは(3.47)で事後分布を計算できる。また、途中、$a_c$を定数Cとおいた。ベータ分布だとわかると正規化項は一意に定まる。)

よって、
\begin{equation}
Beta(\mu | \hat{\alpha}, \hat{\beta}) = C exp((\alpha + \sum_{n = 1}^N x_n - 1)ln \mu + ln (1 - \mu)(\beta + N - \sum_{n = 1}^N x_n - 1)) = C \mu^{\alpha + \sum_{n = 1}^N x_n - 1}(1-\mu)^{\beta + N - \sum_{n = 1}^N x_n - 1}
\end{equation}
これから、(3.54),(3.55)が求まる。

(3.53)に関して、(3.39)より、$h(x_*)=1, t(x_*)=x_*$,(3.51)より、$a_c(\lambda)=-ln \frac{\Gamma(\lambda_1)\Gamma(\lambda_2 - \lambda_1)}{\Gamma(\lambda_2)}$なので、(3.48)に当てはめると、
\begin{equation}
p(x_*|{\bf X}) = \frac{exp(a_c(\hat{\lambda}_1 + x_*, \hat{\lambda}_2 + 1))}{exp(a_c(\hat{\lambda}_1, \hat{\lambda}_2))}=\frac{\frac{\Gamma(\hat{\lambda}_1 + x_*)\Gamma(\hat{\lambda}_2 + 1 - \hat{\lambda}_1 - x_*)}{\Gamma(\hat{\lambda}_2 + 1)}}{\frac{\Gamma(\hat{\lambda}_1)\Gamma(\hat{\lambda}_2 - \hat{\lambda}_1)}{\Gamma(\hat{\lambda}_2)}}
=\frac{\Gamma(\hat{\lambda}_1 + x_*)\Gamma(\hat{\lambda}_2 - \hat{\lambda}_1 + 1 - x_*)}{\Gamma(\hat{\lambda}_2 + 1)}\frac{\Gamma(\hat{\lambda}_2)}{\Gamma(\hat{\lambda}_1)\Gamma(\hat{\lambda}_2 - \hat{\lambda}_1)}
\end{equation}
(3.47),(3.51)を用いて、式をまとめ、(3.54),(3.55)で置き換える。(3.25)にも注意すると、
\begin{equation}
\begin{split}
p(x_*|{\bf X})
=\frac{\Gamma(\lambda_1 + \sum_n x_n + x_*)\Gamma(\lambda_2 + N - \lambda_1 - \sum_n x_n + 1 - x_*)}{\Gamma(\lambda_2 + N + 1)}\frac{\Gamma(\lambda_2 + N)}{\Gamma(\lambda_1 + \sum_n x_n)\Gamma(\lambda_2 + N - \lambda_1 - \sum_n x_n)}\\
=\frac{\Gamma(\alpha + \sum_n x_n + x_*)\Gamma(\beta + N - \sum_n x_n + 1 - x_*)}{\Gamma(\alpha + \beta + N + 1)}\frac{\Gamma(\alpha + \beta + N)}{\Gamma(\alpha + \sum_n x_n)\Gamma(\beta + N - \sum_n x_n)}\\
=\frac{\Gamma(\hat{\alpha} + x_*)\Gamma(\hat{\beta} + 1 - x_*)}{\Gamma(\hat{\alpha} + \hat{\beta} + 1)}\frac{\Gamma(\hat{\alpha} + \hat{\beta})}{\Gamma(\hat{\alpha})\Gamma(\hat{\beta})}
=\frac{\Gamma(\hat{\alpha} + x_*)\Gamma(\hat{\beta} + 1 - x_*)}{(\hat{\alpha} + \hat{\beta})\Gamma(\hat{\alpha} + \hat{\beta})}\frac{\Gamma(\hat{\alpha} + \hat{\beta})}{\Gamma(\hat{\alpha})\Gamma(\hat{\beta})}
=\frac{\Gamma(\hat{\alpha} + x_*)\Gamma(\hat{\beta} + 1 - x_*)}{(\hat{\alpha} + \hat{\beta})\Gamma(\hat{\alpha})\Gamma(\hat{\beta})}\\
\end{split}
\end{equation}
この先は$x_* \in \{0, 1\}$に注意し、ベイズ推論による機械学習(緑本)の(3.19)-(3.21)のやり方を参考にすると\footnote{購入をおすすめします。}、上記の式が平均、$\frac{\hat{\alpha}}{\hat{\alpha} + \hat{\beta}}$となるベルヌーイ分布になることがわかり、(3.53)が導出される。

\end{document}
